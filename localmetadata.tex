\author{Sally~J. Delgado}
\title{Ship English}
\subtitle{Sailors’ speech in the early colonial Caribbean}
\renewcommand{\lsSeries}{scl}
\renewcommand{\lsSeriesNumber}{4}
\renewcommand{\lsImpressionCitationAuthor}{Delgado, Sally J}

\renewcommand{\lsID}{166}
\BookDOI{10.5281/zenodo.2589996}
\renewcommand{\lsISBNdigital}{978-3-96110-151-1}
\renewcommand{\lsISBNhardcover}{978-3-96110-152-8}

\dedication{For Mervyn  Alleyne 1933-2016}
\BackBody{
% \begin{quote}
\textit{In this thoroughly researched and brilliantly written volume, Sally Delgado opens up vitally important new avenues for the study of the role of marginalized peoples such as sailors and convicts in the emergence of creole languages and other contact varieties of the colonial era. Since the ground-breaking work of Ian Hancock some decades ago, we have been waiting for a coherent and comprehensive work such as this to establish a framework and data base for making the systematic investigation of Ship English a reality.}
(Nicholas Faraclas, University of Puerto Rico)
% \end{quote}

\medskip

% \begin{quote}
\textit{The historiography of creole languages has long included frequent references to maritime English with only sketchy indication of just what this kind of speech was like. Sally Delgado has at last provided a comprehensive survey of a dialect that emerged on shipboard among sailors, which became one element in the new Englishes that emerged worldwide amidst the transatlantic slave trade and beyond. Anyone interested in creole languages, as well as those who would like their acquaintance with sailors' speech in the past to get beyond the likes of ``Aye, matey'', should consult this new volume.}
(John H. McWhorter, Columbia University)
% \end{quote} 

\medskip
% \begin{quote}

\textit{While classes on “World English” are increasingly being included in university curricula, they provide little on how that language left the shores of Britain in the first place, and what it was like; until now, research in dialect studies on what was spoken on board ship during the early colonial period has been minimal. Dr. Delgado’s book is the first full-length study to address this; in addition to examining the distinctive characteristics of Ship English as an occupational register, it proposes that as the earliest contact variety, it provided the input in the formation of the Atlantic English-lexifier creoles. A groundbreaking study, essential reading for dialectologist and creolist alike. }
(Ian Hancock, The University of Texas)
% \end{quote}

}

\illustrator{Sebastian Nordhoff}
\proofreader{%
Alena Witzlack,
Andreas Hölzl,
Brett Reynolds,
Carla Bombi,
Carmen Jany,
Daniil Bondarenko,
George Walkden,
Hella Olbertz,
Ivica Jeđud,
Jean Nitzke,
Jeroen van de Weijer,
Laura Melissa Arnold,
Melanie Röthlisberger
}




