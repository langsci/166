\addchap{\lsAcknowledgementTitle}
First and foremost, I would like to thank all the archivists, collections specialists, and librarians at the many libraries and archives I have visited, including the wonderfully helpful and kind staff of the National Archives in Kew, London; the knowledgeable and generous staff of the Merseyside Maritime Museum Archive and Library in Liverpool, England; and all of the volunteers and specialists who gave their time to guide me through the collections at the National Maritime Museum in Greenwich, England. I also thank the many volunteers and specialists at the Barbados Department of Archives, the Whim Archive, the Josefina del Toro Collection, and the National Archives of Trinidad and Tobago who are working tirelessly and often in difficult conditions with little funding to maintain and promote the documents located in collections around the Caribbean. I thank all of these wonderful individuals for their time and patience as they communicated with me on site and at distance about material that was critical to my understanding of this subject, regardless or not as to whether this information made it into the dissertation. 

I would like to acknowledge the many many hours of work that Ann Albueyh, PhD, invested in her work as academic advisor to my doctoral dissertation that gave rise to this book. Her observations, suggestions, and guidance have been invaluable in helping me shape the final product. I am also hugely grateful for all the professional advice and insight she has given me throughout this process. I thank Nicholas Faraclas, PhD, another bedrock of my doctoral academic committee, whose work ethic, worldview and generosity have inspired so much more than my studies. I thank Michael Sharp, PhD, whose insightful comments and input as a reader have been critical in enabling the completion of my doctoral degree after the devastating news of Mervyn Alleyne’s passing in November of 2016. I thank the many professors and support staff at the University of Puerto Rico, Rio Piedras Campus for their motivation throughout my graduate studies and their continued support of my research and professional development. I own specific thanks to the English Department and the Deanship of Graduate Studies and Research (DEGI by its Spanish acronym) for the financial support that their research assistant and teaching assistant positions, travel grants, and scholarship awards have given me throughout my time at the university.

I acknowledge the huge influence that Ian Hancock, PhD, has had on this book. His early work on creole genesis theory first suggested the idea and coined the phrase “Ship English” to describe a variety that was spoken by British sailors in the early colonial context (\citealt{Hancock1976}: 33.) Since I first contacted Ian Hancock with my early ideas in 2011, he has given me valuable critical feedback, shared little-known resources, and offered valuable guidance in the development of my objectives and research plan. I have gained a great deal and continue to benefit from his mentorship and the many hours he has spent communicating with me about a subject that few people have an interest in beyond the acknowledgement of cultural stereotypes. I am very grateful to Ian for his time and particularly for his collaboration on a joint-presentation for the Society of Pidgin and Creole Linguistics that we gave in January of 2017 which also gave me the opportunity to explore his expansive personal library. I hope that this book might assume a humble place among that esteemed collection. 

I thank the all the scholars, educators, and professionals who have permitted me to reproduce images in this book. Specifically, thank you to Gustavo D. Constantino for your clear representation of the mixed methodology research model used in the introductory chapter. Thank you, Mandy Barrow, for all the work you are doing on ProjectBritian.com and your permission to use the map in chapter 3. Thank you to all the professionals at Encyclopaedia Britannica, Inc. for your dedication and your permission to use the map of Atlantic trading winds in chapter 4. Thank you to the curators of the image collections at the National Maritime Museum, Greenwich, London for granting me permission to use two examples of George Cruikshank’s artwork. And thank you to the, still unconfirmed, author of \textit{A} \textit{general} \textit{history} \textit{of} \textit{robberies} \textit{and} \textit{murders} \textit{of} \textit{the} \textit{most} \textit{notorious} \textit{pyrates} (published 1724 in London) for your representation of a mock trial. I also thank historian Marcus Rediker for bringing this image to my attention. 

Last, but perhaps most important of all, I would like to thank my husband and rock, Jose Delgado, for the millions of ways, large and small, that he has supported my research and my writing.  Without his emotional support and his tireless optimism that I could complete what I set out to do, none of this would have been possible. I also thank my mum, Kathleen Dobson, who instilled in me a work ethic and sense of dogged determinism that has been invaluable in the most challenging of weary dust-filled days during my time at the archives. Thank my brother, Nicholas Ruxton-Boyle for his encouragement and rent-free accommodation in London for some of my long-haul archive trips. I also thank my many friends, doctoral students, and faculty at the University of Puerto Rico, Rio Piedras Campus, and the University of Puerto Rico, Cayey Campus who have supported me with their feedback and friendship. Finally, many thanks to my two wonderful boys, Luis and Patrick, who fill my life with love and meaning and have shared their mummy with this research for as long as they can remember. 
