\chapter{{Noun phrases}}

This is the first of three chapters that are linguistic in focus and respond to the research questions on the salient markers and characteristics of Ship English. This chapter on \isi{noun} phrases opens with some general comments on the scope of the data presented and continues with four sections moving from the smallest unit of \isi{noun} composition to the largest constructions of the \isi{noun phrase}. The first section on single-word or bare nouns includes a brief discussion of \isi{phonology}, morphology and lexicon, followed by more focused analysis of \isi{genitive} forms, \isi{plural inflection}, and \isi{noun head} omission. The second section on determiners presents data on number and sequence marking, quantifying mass nouns and articles. The third section on pronouns presents data on personal and \isi{possessive} pronouns, expletives, \isi{indefinite} and reflexive pronouns and gives some details about how relative pronouns are used and omitted from modifying clauses. The last section on \isi{noun phrase} modification presents data on pre- and post-nominal modification and focuses on \isi{present participle} phrases and the specific linguistic constraints of phrases headed with the \isi{participle} “being.” 

\section{{General considerations on scope}}\label{sec:5.1}

The smallest unit of linguistic analysis in this chapter is lexical, starting with the \isi{bare noun} component of the \isi{noun phrase}, and the focus of linguistic analysis is syntactic. Yet that is not to suggest that phonological and morphological features did not feature in the \isi{corpus} nor that there is inadequate material for analysis. The reason that such issues are not dealt with in detail in this study derives from a desire to focus on syntactic issues in the knowledge that this area is least represented in the literature on sailors’ speech (discussed in chapter 2) and in no way implies that research in these other areas is either conclusive or comprehensive. Given that the most extensive research into sailors’ speech to date is \citeapo{Matthews1935} monograph on sailors’ pronunciation in the second half of the \isi{seventeenth century} based on phonetic spellings in ships’ logbooks, the scope of this book does not include work on \isi{phonology}. However, my own research suggests that Ship English has distinctive phonological features and the early findings of that research were presented at the Summer meeting of the Society of \isi{Pidgin} and \isi{Creole} Linguistics in Graz, Austria (\citealt{Delgado2015}). To give a brief overview, notable findings include the realization of front vowels in higher position than anticipated, particularly in pre-nasal contexts, the avoidance of long vowels, and a fricative-plosive interchange that appears to be conditioned by social rather than linguistic factors. However, this book will only present selected discussion of \isi{phonology} as it relates to the linguistic features under discussion and is not intended to represent the wide range of phonological variants found in the \isi{corpus}. 

Morphological issues are also of great interest, yet have been previously addressed, albeit in brief, in \citegen{BaileyRoss1988} article on the morphosyntactic features of Ship English that focuses on evidence of variation in \isi{tense marking} and the \isi{copula}. Like phonological data, selected morphological data in this current \isi{corpus} will be presented only as they apply to the category of speech under discussion, for example, the morpheme “-s” as it applies to \isi{plural inflection} and \isi{third person singular} \isi{verb} agreement, and the morphemes “noon” and “yest” as they show evidence of free and bound variation in sequence marking. I have chosen not to dedicate sections explicitly to morphology, particularly as early indications suggest that further scholarship is needed to analyze potential morphological constructions and constraints. For example, one interesting feature that requires further study would be the use of pre-nominal “a-” in words such as “a board” [ADM 52/1/8,]~“a shore” [ADM 52/1/8], and “a back” [DDB6 8/4] compared to the more recognized pre-verbal usage of the morpheme in phrases such as “a Cruising” [ADM 52/2/6], and “a pyrating” [HCA 1/99 Barbados {1733}] that has parallels with the <a> prefixing denoting \isi{durative aspect} in Appalachian English (\citealt{Hickey2004}: 612; \citealt{Montgomery2001}: 148).\footnote{Further research into the morphology of Ship English might also focus on a preference for nominalization over \isi{copula} or linking verbal constructions such as is expressed in nominal forms using “-ness,” e.g., “the same forwardness” [HCA 1/99/45], “the involuntaryness of his actions” [HCA 1/99/51], and “his unaquaintedness with what path they should follow” [HCA 1/99/52]. Study into the use of \isi{adjectival} superlatives using <est> that also embed into \isi{noun} phrases when they might have been realized in verbal constructions may also prove enlightening, e.g., “was one of the forwardest in robbing her” [HCA 1/99/37], “he was one of the activest and Briskest among the Pyrates” [HCA 1/99/7], and “they were the activest and Leadingest Men among the Company” [HCA 1/99/39.]} Although morphological features such as those suggested here are interesting, this book does not propose to deal with them in full but only aims to highlight their usage with a specific syntactic feature and flag the phenomena for potential future study. 

\section{{Bare nouns}}\label{sec:5.2}

\subsection{{Morphology and lexicon}}\label{sec:5.2.1}

Ship English permits a degree of freedom and variation with morphemes that are explicitly bound in other varieties of English, such as the morphemes “noon” and “yest” that are permitted in both bound and free contexts. The morpheme “noon” only survives in modern English in the bound context of “afternoon” and as a free morpheme meaning midday. Yet, in the \isi{corpus}, the process of nominal compounding using the morpheme “noon” occurs with various referents of time, e.g., “this forenoon” [ADM 52/1/8], “yest noon” [ADM 52/2/5], “this day noon” [ADM 52/2/5], “after last noon” [T/70/1216/12], and “to day noon” [ADM 52/2/5]. This variant usage suggests that the process of compounding happened progressively from a \isi{prepositional phrase} using the free morpheme e.g., “after noon” [ADM 52/1/7] through progressive nominalization with the explicit use of a \isi{definite article}, e.g., “in the after none” [ADM 52/2/1], and finally developed the compounded \isi{lexeme} “afternoon” [ADM 52/2/3]. It is also possible that this lexical change was happening with other free morphemes such as “fore” or “for” (i.e., before) that produced the archaic \isi{lexeme} “fornoon” [ADM 52/2/3] as an antonym to the term “afternoon.” Such variant usage of what we might consider bound morphemes by modern standards is also reflected in the usage of the term “yest” as we know it from the \isi{lexeme} “yesterday.”  In Ship English, evidence indicates that a free morpheme “yest” was used in variation with other time referents, e.g., “yester night” [ADM 52/2/3], “yestday noon” [ADM 52/1/5], “yest noon” [ADM 52/2/5], and “yest afternoon” [ADM 52/2/5]. The variation evidenced with morphemes like “noon” and “yest” might be a manifestation of a wider phenomenon in Early Modern English in which the language had become more analytic favoring the use of free over bound morphemes (\citealt{MillwardHayes2012}) yet might also suggest characteristic diversity among sailors who resisted nominal compounding with markers of time and instead made use of free morphemes with a range of nominal markers.  

Modern linguistic classifications of lexicon are applied throughout this chapter and the following chapters on verbs and larger syntactic constructions, yet discussion of the material in these chapters acknowledges that the variety of Ship English evidenced in the \isi{corpus} was a manifestation of Early Modern English that potentially used lexical items in ways that are no longer acceptable. To illustrate, the \isi{noun} “fortnight” deriving from a contraction of the Old English “fēowertyne niht” which became “fourteniht” in Middle English and “fortnight” by Early Modern English, literally meaning “fourteen nights” (\textit{Oxford Eng. Dict.} 1989, Vol 6: 102) is used according to its etymology as a temporal \isi{noun} in Ship English, e.g., “fort night last” [HCA 1/9/64] and “they were taken about a fortnight afterwards” [HCA 1/13/98]. The \isi{nominal form} was also used in the \isi{predicate noun} position in \isi{copula} constructions, e.g., “munday last was fortnight” [HCA 1/13/97] and “yesterday was fortnight last” [HCA 1/9/67]. Yet the \isi{lexeme} could also be used as an adverbial sequence marker accompanying the \isi{copula}, typically in an inter-verbial position in a passive \isi{verb phrase}, e.g., “was fortnight taken in the said barque” [HCA 1/13/97] and “on Thursday last was fortnight met” [HCA 1/9/67]. In this context, the word is potentially used as a contracted from of a phrase meaning “a fortnight ago” in a prenominal position, yet that fact that the specific \isi{lexeme} “fortnight” can function nominally and adverbially suggests that this was a variant feature of sailors’ usage that was not customary with the nominal etymology of the word. Furthermore, this example might suggest that modern day linguistic typology of word constituents\footnote{The term “constituent” is applied here and throughout this work following Morenberg’s definition as “an individual word or a group of words that fill a single slot” (2010: G-341)} may be inadequate to reflect the variation inherent in sailors’ speech.\footnote{I do not propose that the phenomenon of multifuctionality is unique to Ship English; it is a feature typical of non-standardized varieties.} In addition to such routine vocabulary, Ship English was likely to have used many words that were specific to the technical equipment or movements of the vessel and \isi{crew}, thus forming a kind of \isi{professional jargon}. Indeed, this jargon composes the bulk of the literature on sailors’ speech: the maritime dictionaries and word lists (see §2.1.1). There is potential for future research,\footnote{Further research into the lexicon of Ship English might focus on how such words compare to the lexicon of English-lexifier creoles or language universals given the role of mariners in providing for and settling the European colonies in the sixteenth and seventeenth centuries. Future studies might also find interesting data on how existing lexicon developed alternative denotation in sailors’ speech communities, e.g., “he was only to be \textit{a husband} for them and not to over charge them” [SP 42/6 emphasis added] in which the \isi{lexeme} “husband” denotes an agent appointed by the owners to attend to the ship’s affairs while in port (\textit{Oxford Eng. Dict.} 1989, Vol 7: 510,) and “the \textit{Lizard} bore NbE” [ADM 52/2/3 emphasis added] in which the “lizard” refers to the peninsula of Cornwall seen from the starboard side of a vessel when sailing for Portsmouth or Plymouth. Studies may also focus on the etymology and usage of nautical words such as the word “slatch,” potentially deriving from the word “slack” and dating from 1625 in nautical use to denote a portion of loose rope that hangs overboard or a brief interval of favorable weather (\textit{Oxford Eng. Dict.} 1989, Vol 15: 659,) e.g., “hope...wee may have a slatch of a faire wind” [ADM 106/288/30]. Regionalisms such as the northern term “lads” also feature in Ship English and might be a fruitful focus for future research, e.g., “the young Lads had killed the master” [HCA 1/99/10] and “Give way my lads” [445f.1/41]. However, in the context of this chapter, these few examples are briefly presented only to motivate potential future directions of research by indicating areas of interest in this \isi{corpus}.} but as lexicon is not my central concern, I will limit my observations here to an acknowledgement that Ship English was a variety of Early Modern English and as such, there is potential for certain lexical items to align with or show variations on obsolete usage.

Certain nominalizations speak to the multilingual composition of sailors’ speech communities in that they have either been borrowed or influenced by another language, or show modern-day parallels with other languages. For example, the word “rhumb” e.g., “Clear Cours upon severall Rumbs” [ADM 52/1/7], denotes either a rhumb line or a direction on a nautical chart and dates from the end of the \isi{sixteenth century} (\textit{Oxford Eng. Dict.} 1989, Vol 13: 870). The word seemingly derives from the Spanish “rumbo” (course or direction) and was potentially transferred to English ships via French usage of the rhumb line in navigational practices. Another word adapted from Spanish was “plate,” meaning money and deriving from an anglicized form of the Spanish word “plata” (silver). Words such as “rhumb” and “plate,” were adapted from Spanish but other words were borrowed without adaption, e.g., “commanded the Soldadoes” [HCA 1/98/265] “making the best of our way to windard to weather Disseado but gained noe Ground” [ADM 52/1/7]. The two words “Soldadoes” (soldiers) and “Disseado” (i.e., \textit{deseado} or desired) in the previous citations do not appear to be anglicized, and furthermore, the second example “weather Disseado” uses Spanish post-nominal modification rather that the anticipated order of the \isi{adjective} preceding the \isi{noun} in standard English: “weather desired” suggesting that the speaker was familiar with Spanish speech and not just Spanish words. Other phonological evidence supports the suggestion that sailors had influence from other languages when speaking English. For example, the loan of the word “Espaniols” [HCA 1/99] to refer to Spanish sailors demonstrates the Spanish-language phonotactic constraint that a syllable may not begin with /s/ followed by a consonant. And if such contexts manifest themselves, e.g., as in the combination /sp/ in words such as “spy,” then the Spanish phonotactic constraint dictates that the /s/ phoneme is assigned to the prior syllable in an obligatory process of epenthesis that adds /e/ before the initial /s/ to create an additional syllable (\citealt{Schnitzer1997}: 85). The word “Espaniols” may also have been preferable to “Spaniards” as it avoids the three-part consonant cluster in the coda of the last syllable “-rds” that features in the Standard English vocabulary. Although orthography suggests that the speakers of Ship English did not default to Spanish phonotactic constraints in general, compelling evidence of this type of epenthesis is suggested by the spelling of the word “spy” in archival documents, e.g., “he espyed a Vessell there riding” [HCA 1/9/67] “they espied a vessell” [HCA 1/9/6], “they espied a boat” [HCA 1/99 \isi{Bahama} Islands 1722] and “espied a saile and chased him” [HCA 1/9/13.]\footnote{The English use of the \isi{verb} “espy” derives, in part, from a \isi{verb} form in Old French “espier” dating back to around 1250 that was transferred into Middle English and then potentially reinforced by cognates from other Romantic languages (e.g., Spanish, Portuguese, and Italian) and Germanic languages (e.g., Dutch, Swedish, and Old Norse,) (\textit{Oxford Eng. Dict.} 1989, Vol 16: 383).} Thus, not only lexical, but also syntactic and phonological evidence suggests that \isi{language contact} in the \isi{speech community}, and specifically contact with Spanish, may have manifested itself in Ship English through nominal borrowings and potential phonological and syntactic interference. ~

\subsection{{Genitives}}\label{sec:5.2.2}

Further evidence that English-speaking sailors favored syntactic constructions common to Romance languages can be found in the analysis of genitives (typically reflecting possession, partition or agency). Before \isi{language change} in the Middle English and Early Modern English period developed new ways to mark the \isi{genitive case}, Anglo Saxon use of \isi{uninflected} genitives was commonplace. Although this was certainly not common in the period under study, there are examples of such archaic constructions in the \isi{corpus}, e.g., “under Holland colors” [HCA 1/10/2] and “the King of Ennglande pape[r]” [CO 5/1411/78]. However, more commonly, the \isi{corpus} shows examples of the two forms of \isi{genitive} marking still permitted in modern standard English: either a \isi{noun} followed by an apostrophe and an “s” morpheme that combines to form a \isi{genitive} \isi{noun}, or a \isi{noun} appearing after the \isi{preposition} “of” in a \isi{prepositional phrase} that is \isi{genitive} in function. ~The linguistic data in the \isi{corpus} showed that both forms were available in Ship English, yet, the use of the contracted form apostrophe plus “s” was more unusual. This might attest to the fact that the “-’s” \isi{possessive} form was a later development in the Early Modern English period that was still in competition with the Anglo-Saxon use of \isi{uninflected} genitives or the Latin use of prepositional \isi{genitive} phrase during the period (\citealt{MillwardHayes2012}: 266). Perhaps due to the fact that this variant was more recent, its pronunciation was still variable and so sailors may have interpreted the final [s], [z], or [Iz] allophone of an inflected \isi{genitive} \isi{noun} as a contraction of the \isi{possessive} \isi{pronoun} “his” rather than an inflectional ending, particularly if the \isi{noun} already ended in a sibilant and the /h/ of the following word was unstressed, e.g., “in Roberts his Company” [HCA 1/99/170], “sailing under Robert’s his Command” [HCA 1/99/170], “Roberts his Death” [HCA 1/99/51], and “Robert Clarke Capt Hobbs his Servant” [HCA 1/9/51] which are more likely to have been intended as “in Roberts’s company,” “sailing under Roberts’s command” “Roberts’s death” and Robert Clarke, Captian Hobbs’s servant.\footnote{Millward \& Hayes suggests that this type of orthographic misinterpretation occurred on a wider scale in the Early Modern English period (2012: 260).}” Yet, overwhelmingly, the linguistic data in the \isi{corpus} showed that the default form of expressing \isi{genitive case} was through the use of a \isi{prepositional phrase}. Examples showed this form was used: to indicate possession, e.g., “the luggage of his majties Embassador” [HCA 1/98/271]; to indicate source composition, e.g., “a very hard gayle of wind” [HCA 1/101/473] and “a sudden Storme of wind” [HCA 1/14/107]; to indicate partitive relationships, e.g., “What troopers of horse” [HCA 1/9/105], “the high court of admiralty” [HCA 49/98/106], “the master of the examined” [HCA 1/52/1], “they of the \textit{Sea Flower}” [HCA 1/53/57]; and also to show appositive relationships, e.g., “the River of Thames” [HCA 1/9/64] and “the bay of Chesepeak” [HCA 1/99 \isi{Williamsburg}, Aug 14 1729]. It appears that the use of the contracted form “-’s” was not favored in Ship English, and although this form was universal throughout Early Modern English period, it seems that sailors may have preferred to mark \isi{genitive case} with prepositional phrases, specifically because this construction aligned with Spanish, French and potentially other languages that contributed to the linguistic diversity on board ships and reduced the number of variations in cognitive processing.

Genitive case marking using \isi{possessive} pronominal determiners is common in Ship English, although this sometimes resulted in \isi{double genitive marking}. Double \isi{genitive} marking, or \isi{genitive} concordance, occurred when a pronominal \isi{possessive} determiner such as “my” “his,” or “her” was used in a \isi{prepositional phrase} “of…” that also marked \isi{genitive case}, e.g., “some vessel of his” [HCA 1/12/4], and “these few lines of mines\footnote{The pluralization of the first person \isi{possessive} \isi{pronoun} “mine” also potentially reflects Spanish morphology i.e., “mio” (sing) “mios” (pl).}” [HCA 1/99 loose letter c. 1730] in which the \isi{genitive} is marked once by the \isi{prepositional phrase} headed by “of” and secondly by “his” and “mines” respectively. This construction is most common in the \isi{third person} form\footnote{Note that examples using the female \isi{third person} \isi{possessive} pronominal determiner “her” are debatable as the \isi{accusative case} “her” is identical to the \isi{genitive} form, yet they are treated here as representative of the \isi{genitive} form given other evidence suggesting this construction.} , e.g., “the Comand of her” [T 70/1/11], “Comander of her” [HCA 1/14/17], “the Second \isi{Mate} of her” [HCA 1/99/144], “Carpenter of her [HCA 1/99/153], and “the Master of her” [HCA 1/99/39]. It may be that references to rank such as this were idiomatic and that the \isi{genitive} concordance consisting of using a \isi{prepositional phrase} in conjunction with a \isi{possessive} pronominal determiner was considered correct usage, as evidenced by the witness testimony “the Master and \isi{Mate} \textsuperscript{of her} were knocked over board with the Boom at Sea” [HCA 1/99/11] in which the words “of her” are inserted superscript, presumably after the original was composed and later revised for corrections.  This use of double genitives in Ship English does not appear to follow Peters’ claim that such constructions may only be applied to human referents (2007: 162) because the “her” of the previous citations refers to the vessel itself and not a female human. However, this is less problematic in consideration of the maritime custom of referring to the ship (and often naming the ship) as a woman, e.g., “we suposed her [a sighted ship] to be standing the saime Course” [DDB6 8/4]. The gendering of sea-going vessels is explored in Creighton and Norling’s \textit{Iron Men, Wooden Women} (1996,) including specific details about how, in the \isi{seventeenth century}, wooden sailing vessels were often gendered female owing to English medieval customs of naming a vessel for the monarch’s mistress and referring to the antiquated custom of sailors invoking a deity of the sea - often a woman. Therefore, in the context of maritime culture, it is more understandable that a double \isi{genitive}, thought to be confined to usage with human referents, is applied to (female) sea-going vessels. 

Although \isi{genitive} concordance often occurred with the \isi{third person} female pronominal \isi{genitive} “her” used in “of…” phrases marking \isi{genitive case}, the same structures are not as common with male or \isi{plural} referents. Instead, and even when \isi{genitive case} was clearly implied, the \isi{accusative case} of the \isi{pronoun} was preferred, e.g., “him wife” [HCA 1/9/8], “him lights” [ADM 51/4322/4] and “the goods of him” [ASSI 45/4/1/135/5]. There are no examples of \isi{third person} \isi{plural} \isi{possessive} pronouns, either “their” or “theirs” in double \isi{genitive} constructions in the \isi{corpus}. Examples of constructions using a \isi{third person} \isi{pronoun} referent make use of the \isi{accusative case} in prepositional phrases, e.g., “pyracies that have been committed under Colour of them” [HCA 1/99/10] and “the pyrate and his consort two ships of them” [HCA 1/99/39]. Thus, we might surmise that it was specifically female pronominal determiners that caused \isi{double genitive marking} as they combined with the default variant of marking \isi{genitive case} with prepositional phrases rather than nominal inflection. 

The linguistic context that may have prompted the use of variant forms of genitives in Ship English are extremely difficult to derive, more than anything because of the limited number of examples available upon which to base a satisfactory interpretation. However, different forms of marking \isi{genitive case} sometimes appeared in close proximity in the written records and in documents prepared in the same hand, implying that they were potentially in competition and maximally variable in the speech of individuals rather than regionally or socially distributed. For example, one \isi{witness deposition} taken on 28 March {1722} reads, “this man as Carpenter of her, and when brought on Board the \textit{Fortune} Carpenter’s mate going on Board” [HCA 1/99/153] and includes examples of the Anglo Saxon \isi{uninflected} form “\textit{Fortune} Carpenter,” the prepositional \isi{genitive} phrase “Carpenter of her” and the bound “s” morpheme “Carpenter’s mate” within the same utterance. Similarly, another deposition taken on 14 August {1729} includes the clause “prisoners took away a new jacket of his mans from his back” [HCA 1/99 \isi{Williamsburg}, Aug 14 1729] and includes examples of the prepositional \isi{genitive} phrase “jacket of his mans,” and a suggestion of the bound “s” morpheme in the use of the word “mans” although it is not represented orthographically with an apostrophe, it clearly does not refer to the \isi{plural} “men” but rather a \isi{possessive} form denoting that the jacket belonged to “his man” i.e., his servant. ~Although few, such examples show that even by the early decades of the \isi{eighteenth century}, there was no universal default \isi{genitive} marker but rather that the historic and contemporary variants available to each speaker were used concurrently, even within the same \isi{noun phrase}. 

\subsection{{Plural inflection}}\label{sec:5.2.3}

The \isi{corpus} includes repeated examples of nouns that are pluralized by numerical determiners but do not inflect with an “s” morpheme, specifically regarding units of measurement. Logbooks and witness testimony frequently refer to the number of fathoms\footnote{A nautical unit of length, 6 feet or approximately 1.8 meters, used to measure the depth of water and often measured with a sounding or lead line dropped over the side of the ship.} that a vessel measured, and most of these entries include a phrase in which the single form of the \isi{count noun} “fathom” is prefaced with a \isi{cardinal number}. Sometimes the singular form of the \isi{count noun} is prefaced with a number that is written out, e.g., “five fathome” and “sevean fathome water” [both citations from HCA 1/9/155]. However, more commonly, authors expressed cardinal numbers in numerical form, e.g., “anchd in 7 fathm” [CO 5/1411/675] and “had 9 fathom” [ADM 52/1/7]. It is interesting that many examples of usage include some variant spelling of the fixed expression “fathom water,” e.g., “dropping our anchar in 6 fatham water” [ADM 52/1/6], “sevean fathome water” [HCA 1/9/155], “12 fathom Water” [ADM 51/3797/1], “30 fadam water” [T/70/1215 Oct 15 entry], “in 33 Fathom-water” [1045.f.3/1/16], and “had 50 fathome water” [ADM 52/3/12]. This expression suggests that the phrase derives from an underlying construction including a pre-article composed of a number and “fathoms of” followed by the \isi{bare noun} “water” whose contracted form is understood among speakers and recipients. Yet, although the use of the \isi{uninflected} \isi{noun} “fathom” predominates in \isi{corpus} examples, there is evidence of \isi{free variation}, e.g., “wee had som time 7 fatham and 3 fatham…[and other times] 5 and 6 fathams” [DDB6 8/4]. The fact that both the singular form “fatham” and the \isi{plural form} “fathams” are used in the same document by presumably the same author implies that both variants were available to speakers and could be used within the same utterance. 

Many of the other units of measurement that are demonstrated to be \isi{plural} with numerical determiners but without using inflected \isi{noun} forms also relate to nautical calculations of time, distance and weight. Examples of units of time expressed with an \isi{uninflected} \isi{bare noun} include “seaven night last past” [HCA 1/9/63] and “in few day after” [HCA 1/52/75] contrary to the expectation that bare nouns would take the \isi{plural form}, i.e., “nights” and “days” after a determiner of quantity. Measurements of distance expressed in the unit of length composing 12 inches often used the singular form of “feet,” e.g.,~“about 2 foot in heighth” [1045.f.3/1/25], “3 or 4 Foot high” [1045.f.3/1/27], and “several Foot of Water in the hold” [HCA 1/99 \textit{The American: Weekly Mercury} No.618, Oct 28-Nov 4 1731]. Distance expressed in nautical miles also used the singular form of the \isi{bare noun} in collocation with numerical determiners e.g., “Laguna was but 3 Mile off” [1045.f.3/1/18], “Dist[ance] 196 mile” [ADM 52/1/11], and “up in the Country 15 mile” [T/70/1216/8]. Weight measurements showing use of \isi{uninflected} nouns refer to tons and pounds of cargo, e.g., “Burthen about two hundred and fifty ton” [HCA 1/52/103], “got 7 or 8 Tun of Salt” [1045.f.3/1/30], and “Butter...332 pound, Suffolk Chefe 375 pound, Bread in two baggs 179 pound \& Rapines 113 Pound” [ADM 52/1/6]. The use of \isi{uninflected} “pound” as a unit of weight is also reflected in its use as a unit of currency, e.g., “Fifty pound in mony… and for fifty pound more” [HCA 1/14/167]. The linguistic tendency in Ship English to maintain \isi{uninflected} bare nouns after a determiner of quantity was not unique to sailors’ language however, Millward \& Hayes explain that “measure words like mile, pound, fathom, pair, score, thousand, and stone frequently appeared without a pluralizing -s, especially after numerals” throughout the Early Modern English period (2012: 167).

It certainly may have been that the tendency to retain unmarked plurals after a cardinal determiner in Ship English reflected wider Early Modern English usage at the time, yet sailors’ use of non-traditional and figurative units of measurement to refer to the size and capacity of their communities marked Ship English as distinctive. The number of guns that a ship could carry was often used as a measurement of size, and although there are numerous references to the inflected form of this word in the \isi{corpus}, there are also a few examples of its \isi{uninflected} usage with a numerical determiner, e.g., “this ship to have 20 or 24 gun” [T/70/1216/13]. Sometimes units of measurement to count the number of vessels in a company, fleet, or \isi{convoy} were expressed with singular nouns, e.g., “mett with three East India Shipp” [ADM 52/1/1]. However, the literal unit of measurement “ship” was most frequently inflected, e.g., “two ships” [HCA 1/99/105] and “severall ships” [ADM 52/3/7]. Much more common in the \isi{corpus} were the frequent examples of \isi{uninflected} figurative units of measurement to count the number of vessels, specifically the synecdotal use of the singular \isi{noun} “sail” to refer to a vessel, e.g., “3 sayle more” [ADM 52/2/5], “twenty Sayle of Ships” [ADM 52/3/12], “20 sayle of Merchant Shipps” [ADM 51/4322/1], and “20 Saile of third rates” [ADM 51/4322/4]. The last three examples that include the inflected nouns “ships,” “Shipps,” and “third rates” respectively suggest that the use of the \isi{uninflected} “sail” was specific to the pre-article composed of a \isi{cardinal number} and “saile of” followed by an inflected (and literal) \isi{noun} such as “ships” drawing comparisons with the idiom “fathoms of water” previously discussed. Yet even when not used as a pre-article,\footnote{The term “pre-article” is used per \citet[76]{Morenberg2010} and includes several word classes including partitives, quantifiers, multipliers, and fractions that occur before articles or possessives.} it seems that the \isi{lexeme} “sail” was \isi{uninflected} in the context of its use as a unit of measurement, as illustrated by the following two examples in which the figurative unit of measurement “sail” is not inflected yet the literal units of measurement “ship,” “sloop” and “leagues”~are inflected: “being in number as above 22 sail with 3 \isi{merchant} Ships \& Sloopes” [ADM 52/1/7], and “having discovered four saile about four leagues ashore\footnote{The inflected \isi{plural} “leagues” in this second quotation appears to be an exception to the tendency to use \isi{uninflected} nouns with cardinal determiners in Ship English. The word is inflected even in contexts where speakers use unmarked plurals for other units of measurement, e.g., the \isi{deponent} who refers to “13 fathome...20 fathome...26 \& 27 fathome” but in the same \isi{speech act} also says “7 Leagues” [ADM 51/3/12.]}” [HCA 1/9/155]. In further support of this interpretation, there are no examples in the \isi{corpus} of the \isi{uninflected} use of the word “sail” in its literal sense to refer to the \isi{plural} canvas sheets, instead, when used literally, the \isi{noun} “sail” takes a \isi{plural inflection}, e.g., “with Keept topsailes” [ADM 52/2/9]. Furthermore, this distinctive feature of sailors’ speech was salient enough to feature in published sea-songs of the \isi{seventeenth century}, e.g., “Beset with five sail of Pirates” (cited in \citealt{Palmer1986}: 50) and “Nine sail of ships” (cited in \citealt{Palmer1986}: 65). Thus, we can surmise that speakers of Ship English used \isi{uninflected} \isi{plural} units of measurement that were specific to the \isi{speech community} in addition to idioms that included \isi{plural} nouns marked by cardinal determiners but not inflection. 

Nouns inflected for \isi{plural} marking and used for generic referents are common in Ship English, but examples also show that \isi{uninflected} nominal forms without an article could be used to refer to a generic referent. The phenomenon seems to have been particularly applied to turtles (both the animal and its meat) e.g., “sent out Long boat a shore to Cath Turtle” [T/70/1216/8], and “we liv’d on Goats and Turtle” [1045.f.3/1/11]. The use of the \isi{uninflected} \isi{noun} “turtle” in a \isi{noun phrase} in which the word is correlated with the inflected \isi{noun} “goats” using the conjunction “and” suggests that the \isi{lexeme} “turtle” is an irregular \isi{plural} like “sheep” or “deer” that may not have developed a regular inflected form yet, although later the regularization of the \isi{plural form} adopted the morpheme “-s” to align with the regular pluralization paradigm. Considering the word “fish” and the traditional \isi{uninflected} \isi{plural form} “fish” which is now accepted in addition to the newer and regularized inflected form “fishes,” the suggestion that the word “turtle” was also an irregular \isi{plural} appears more plausible. This interpretation furthermore appears to be supported by usage of the word in contexts which are clearly marked for plurality, such as in a position after a \isi{cardinal number} signifying a \isi{plural} referent, e.g., “bringing 5 small Turtle” [T/70/1216/8]. Hence, although examples such as “turtle” might suggest that \isi{uninflected} nouns were acceptable for generic reference, it is perhaps more likely that newly introduced words (given that turtles are not endemic to the waters around Great Britain) were undergoing a process of regularization that had not yet been fully realized\footnote{The \citet{oed1989} states that the word “turtle” was explicitly “a corruption, by English sailors, of the earlier ‘tortue’” derived from “tortoise” of French origin that referred firstly to the species (from 1657) and later to the flesh of the species (from 1755). (Vol 18: 722).}. 

\subsection{{Noun head omission} }\label{sec:5.2.4}

Nominals in subject positions, a requisite established by the end of the Middle English period (\citealt{MillwardHayes2012}: 274,) can be omitted in Ship English. Human noun-phrase subjects can be omitted when the context of the utterance renders the reference to the agent of the action redundant, either because it has been previously established or is obvious from context. In court depositions, singular human subjects of a clause predominantly refer to the speaker (the witness) or the accused, and \isi{plural} subjects of a clause predominantly refer to the ship’s \isi{crew} or the port authorities. Given so few variants, and considering the context of testimonial speech acts in which the referent is understood from the context of the testimony, witness depositions often omit \isi{noun phrase} subjects, e.g., “Why he did not goe in her [I, the witness] do not well know” [T 70/1/12], “[I, the witness] was to moore him there” [ASSI 45/4/1/135/4 1650], “[I, the witness] can not tell” [CO 5/1411/640], “[he, the accused] Signed and Shared but never fired a gun at the Swallow” [HCA 1/99/92], “[he, the accused] Did not see any application” [HCA 1/99/129], “Fogg comes on So thick [they, the \isi{crew}] had Much trouble” [ADM 52/2/9], “[they, the \isi{crew}] “burnt a towne called Meofe because the inhabitants would not come downe to traffick with them [HCA 1/53/10], “[they, the \isi{crew}] Carried aboard the enemy” [HCA 1/9/105], and “[they, the port authorities] have given him a receipt for them” [T 70/1/10]\footnote{Although these examples may appear debatable out of context, they are sampled from larger legible documents in which the division of utterances are evident from context, e.g., “and upon the Ethiopian coast [nominative missing] burnt a towne called Meofe because the inhabitants would not come downe to traffick with them” [HCA 1/53/10]} . The fact that these omissions reflect spontaneous speech rather than the composition of the \isi{court clerk} are potentially reflected in one deposition that reads “was killed that \textsuperscript{he} thereupon went to the other” [HCA 1/99/4] in which the nominative \isi{pronoun} “he” is inserted superscript, presumably as a correction or clarification after the original utterance was transcribed. 

Nominal subject omission is evident in logbook entries when the human subject of a clause refers to the \isi{crew} in general terms, e.g. “this morning with Snow \& Sleet [we, the \isi{crew}] Struck yards \& topmasts at 5 this morning” [ADM 52/2/3] and “The fogg being Cleared Up [we, the \isi{crew}] Came to Saile” [ADM 52/2/9]. In addition to redundant human subjects, logbooks also show evidence of non-human \isi{nominal subject} omission when the referent is obvious from context. Examples of such omission most notably relate to the weather, and more specifically the wind, e.g., “very fresh \& hard [wind] from the SW” [ADM 52/1/6] and “this 24 howrs [wind] blew hard and hazey weather winds” [ADM 52/2/9.]\footnote{It is also possible that the \isi{noun phrase} “weather winds” that ends the sentence could be the subject and so reflect an inversion of anticipated word order from Subject-Verb (Adverb) to (Adverb) Verb-Subject.}  Similar to the conditions that prompt human subject omission in court documents, the reason for the omission of the non-human \isi{noun} “wind” is most likely due to the redundant nature of the referent when most logbook entries were expected to open with a report on wind conditions, and also the redundancy of the word “wind” when used in collocation with “gusts” and the \isi{verb} “blow” which make the meaning clear. Furthermore, the omission in this context is also potentially motivated by the abbreviated style of writing permitted in logbook entries. In sum, both court depositions and logbook entries show evidence of the omission of \isi{noun phrase} subject heads when the referent is understood from the context or the customary format of the \isi{speech act}, and the fact that these omissions index shared knowledge is not surprising given that the sailors who composed these speech acts were addressing a very specific, small audience.

Certain linguistic constraints condition the omission of \isi{subject noun} heads. The speaker’s inclusion of attendant circumstances, specifically in the form of a fronted \isi{participle phrase} appears to promote \isi{noun} subject omission in the subsequent clause. Examples from witness depositions include: “Nott willing to venture our sailes near any factory And unwilling to keep any to brood fachons amongst us [we, the \isi{crew}] have in the \isi{long boat} turned to sea all such as were unwilling to stay” [HCA 1/14/206], “the sea growing very high [we, the \isi{crew}] were forced on a reife of sand and [I, the witness] was forced to cut away our main mast” [HCA 1/12/2], and “Being asked by the Prisoners from what Post the Spanish vessell Came, and whether they had a Commission from the Spanish King, And in what manner the Vessell was fitted out, [he, the accused] Says the Spansih Vessell Came from Porto Prince in Cuba” [HCA 1/99/5]. This construction of a fronted \isi{participle phrase} giving attendant circumstances followed by a clause with an omitted \isi{noun phrase} subject is also repeated in journal entries, e.g., “I having not hove the grapling, [he, a \isi{sailor}] turns me about, saying, What’s the matter?” [445f.1/43]. The construction is also evident in logbook entries, e.g., “Sunday at 7 in the morning weighd wth a fresh Gale, and got into the Gulf stream, but the weather being squally [we, the \isi{crew}] could not hold it, so [we, the \isi{crew}] were forced to bear up and anchor in 7 fa water” [ADM 52/2/6]. The repetition of this construction in court depositions, journal and logbook entries suggests that omitted nominal subjects after attendant circumstances may have been a widespread feature of sailors’ speech. 

In addition to the presence of fronted \isi{participle} phrases that appear to permit subsequent \isi{nominal omission}, there is evidence that fronted prepositional phrases and adverbial phrases also permit the subsequent omission of \isi{noun phrase} subjects. Court depositions include frequent examples of fronted prepositional phrases that permit subsequent \isi{nominal omission}, e.g., “after a fight of abt 3 quarters of an howr [we, the \isi{crew}] board” [HCA 1/53/3], and “at sevin [I, the witness] got into the fleet and was ordered to go” [ADM 52/1/1]. And this feature is also evident in logbook entries, e.g., “From yesterday Noone to the no one [we, the \isi{crew}] have had a moderate gaile” [ADM 52/2/8], and “At 4 in the Afternoone [we, the \isi{crew}] came to an anchor” [ADM 52/2/6]\footnote{Citations of use here do not imply that the feature was universal or even consistent for individual sailors, for example, the same author of this logbook entry “At 4 in the Afternoone [we, the \isi{crew}] came to an anchor” later writes, “at 4 in the morning we off weighed” [ADM 52/2/6] showing omission of the \isi{nominal subject} after a \isi{prepositional phrase} in the first example and the inclusion of a \isi{pronoun} in the second.} .  In addition to these examples of fronted prepositional phrases, court document also show that fronted adverbial phrases similarly permit subsequent \isi{nominal omission}, e.g., “Ever since [I, the witness] hath been on St. Marys” [HCA 1/98/256], “after a fight of abt 3 quarters of an howr [we, the \isi{crew}] board” [HCA 1/53/3], and “when they had bin a fortnight or 3 Weeks att sea [we, the \isi{crew}] mett with a ship” [HCA 1/12/4]. Thus, whether the fronted phrase is a \isi{participle} construction, a prepositional construction, or an adverbial construction, it appears that when the attendant circumstance is moved to the start of the utterance, it conditions circumstances that permit subsequent \isi{nominal omission} in the \isi{matrix clause}. 

Clauses that compose correlative constructions yet do not necessarily include a correlative conjunction also permit the omission of a repeated \isi{noun phrase} subject in the second of the two clauses. These clauses are best interpreted to derive from an underlying form using the correlative conjunction “and” which precedes the omitted \isi{noun} subject and are thus represented as such in the following examples, “he was taken...against his will, [and he] had a Wife and 3 Children” [HCA 1/99/95], “he says himself he put the Match to the Gun but that it did not go off, [and he] was taken in John Tarton about 5 Mos agoe” [HCA 1/99/167], and “she Fled into the Bushes. [and she] Knows that the two Sloops were one Destroyed and the other Taken, together with her Husband”~[HCA 1/99 New Providence 1722]. It may be that the use of correlative conjunctions permitted \isi{noun phrase} omission in \isi{direct object} as well as subject positions, e.g., “brought up the money upon deck and divided [it, the money] amongst the Crew” [HCA 1/99/8]. Although examples of object omission were much less notable in the \isi{corpus}, it is possible that the omission of nominal and pronominal \isi{accusative} forms in \isi{object position} were also conditioned by underlying correlative constructions. 

Object \isi{noun} phrases may also be omitted in the \isi{direct object position} and when they appear as the object complement or the object of a \isi{preposition}. Direct objects are omitted in witness statements, e.g. “the Murderers…threatened to Put to Death [those people] who should refuse to take it” [HCA 1/99/8] and “they bind them and every [one] of them” [HCA 1/9/7], and they are also omitted in logbooks, e.g., “took in ten thousand [unit measurement] of wood” [ADM 52/2/3]. The last two examples also suggest common omission of \isi{noun} heads functioning within the pre-article determining phrase of the direct objects. Yet omission is not restricted to this context. Nouns may be omitted when they function as the object complement, e.g., “there was no wind it was a calme [night]” [HCA 1/9/155], and more commonly when they function as the object of a \isi{preposition}, e.g., “a Small Hoy with petty warrant [officers]” [ADM 52/2/3], “his master having often sent him out on Privateering [voyages]” [HCA 1/99/8 New Providence 1722], and “hoping you are in good [health]” [HCA 1/101/553]. Furthermore, the omission of all three \isi{noun} phrases after an \isi{adjective} in the three previous examples shown appear to suggest that \isi{nominal omission} may have been conditioned using an \isi{adjectival} modifier in addition to the possibility of conditioning by the presence of pre-articles. 

\section{{Determiners}}\label{sec:5.3}

\subsection{{Deictic function}}\label{sec:5.3.1}

Words with a \isi{deictic function}, commonly realized in English with the four \isi{demonstrative} determiners “this” “that” “these” and “those,” showed some variant usage and formation in Ship English. Both the singular and \isi{plural} demonstratives are used with atypical nominal and verbal agreement in Ship English, e.g., the expressions “this Dutch Interloping Ships” [HCA 1/99/105] and “these lines is to arkquint you” [HCA 1/52/51], used in a \isi{witness deposition} and a personal letter, respectively, show variant nominal and verbal agreement in terms of the singular or \isi{plural} nature of the \isi{subject noun} phrase. In the first example, the singular \isi{demonstrative} “this” refers to the \isi{plural} \isi{noun} “ships” and in the second example, the \isi{plural} \isi{demonstrative} \isi{noun phrase} “these lines” is followed by the \isi{copula} \isi{verb} conjugated to a \isi{third person singular} subject. In other examples, redundant demonstratives are in competition with other determiners, e.g., “but these their good designs were discovered” [HCA 1/99/80] in which the determiner “these” and the \isi{pronominal form} “their” compete in the determiner position. The formation of \isi{demonstrative} determiners from \isi{accusative} pronominal forms was also a notable feature in the \isi{corpus}, e.g., the word “them” in the excerpts “don’t you see says he them two ships” [HCA 1/99/105], “he was one of them Pyrates” [HCA 1/99/105] and “t’was too good for them people” [HCA 1/99/110.]\footnote{This use of \isi{accusative} pronouns with a \isi{deictic function} in a \isi{pre-nominal position} is still a feature of certain vernacular dialects of English.} Thus, although only limited examples of determiner use show in the \isi{corpus}, they suggest that \isi{demonstrative} number agreement was not universal, that deictic markers were permitted to compete with other determiners in \isi{pre-nominal position}, and that \isi{accusative} pronouns could take a \isi{deictic function} when used in a \isi{pre-nominal position}.  

\subsection{{Number marking}}\label{sec:5.3.2}

Post nominal lexemes and \isi{indefinite} pre-articles denote estimated quantities. Although logbooks made extensive use of symbolic number marking specific to maritime shorthand and witness depositions included roman numerals (specifically to denote ages of deponents] these are not discussed here as they appear only in the very specific contexts of nautical and court record-keeping conventions and do not appear to have phonetically realized forms distinct from the ordinal or cardinal variants here discussed.  The use of the \isi{lexeme} “odd,” specifically in combination with a round \isi{cardinal number} marks estimated quantities, e.g., “70 odd men” [CO 5/1411/636], “two hundred sixty odd points” [HCA 1/53/57], and “One hundred and odd pieces” [HCA 1/9/58]\footnote{The \citet{oed1989} explains that this usage renders the \isi{lexeme} “odd” as a rare type of \isi{indefinite} \isi{cardinal number} which denotes an unspecified number of lower denomination than the round number preceding it and this usage dates as far back as the fourteenth century (Vol 10: 698).}.  However,~\isi{indefinite} pre-articles in a prenominal position more commonly denote estimated quantities in the samples of Ship English contained in the \isi{corpus}. The word “several” occurs frequently with inflected plurals as we might anticipate from its usage in Modern English, e.g., “we have made severall trips” [ADM 52/1/7], “severall vesells” [ADM 52/3/12], “severall parcells of hulks” [T/70/1216/8], “Severall pasengr boates” [HCA 1/52/88], and “Severall arrived men” [HCA 1/10/2]. Although all the examples above and many more in the \isi{corpus} precede \isi{plural} nouns, the pre-article “several” might also take an \isi{uninflected} \isi{noun}, e.g.,~“we fired Severall Shott” [ADM 52/1/7] and “several Foot of Water” [HCA 1/99 \textit{The American: Weekly Mercury} No.618, Oct 28-Nov 4 1731]. The pre-articles “many” and “some” might also precede \isi{uninflected} nouns, e.g., “and by all the many report” [ADM 106/288/36] and “saw Some Shipp” [ADM 52/1/1], and other quantifying pre-articles also showed this trend, e.g., “a pair of shoe” [HCA 1/99/6], and “a few more shot” [HCA 1/99/14]; yet there are fewer examples of these constructions than the more common phrases with inflected nouns or \isi{noun} phrases. The majority of bare pre-articles take inflected plurals, e.g., “many Moors Shipes” [HCA 1/98/24,]“so many ships being before me” [D/Earle/3/1], “some days in august last” ~[HCA 1/99 New Providence 1722], and “for some weeks past” [HCA 1/99 \textit{The American: Weekly Mercury} No.618, Oct 28-Nov 4 1731]; and the majority of pre-articles using a prepositional particle “of” also take inflected plurals, e.g., “two pairs of large tops” [HCA 1/9/67], “5 pairs of small pearls colored silk tops” [HCA 1/9/67], and “severall parcells of hulks” [T/70/1216/8]. Yet, as previously identified with words of nautical measurement such as “sail” and “fathom,” when such nominals are used as part of a quantifying pre-article phrase, they may not be inflected for \isi{plural} marking when there is a \isi{cardinal number} preceding the pre-article, e.g., “60 Sayle of men of war” [ADM 52/2/5], “2 pair of Pistols on” [HCA 1/99/157], “nineteen paire” [HCA 1/9/67], and “Ninety head of cattle Bulls and Cowes” [HCA 1/52/10]. In sum, although it is possible for quantifying pre-articles to take \isi{uninflected} nouns and include \isi{uninflected} nominal forms in phrasal constructions with the particle “of” (and this was more likely when they denoted a maritime unit of measurement) \isi{uninflected} nouns do not appear to be a grammatical projection of \isi{indefinite} quantifying pre-articles but rather a conditioned or free variant. 

Explicit number marking in Ship English predominantly makes use of cardinal rather than ordinal numbers in pre-nominal positions\footnote{Although there is a suggestion that cardinal numbers could occur in \isi{post-nominal position}, e.g., the line “They had not sailed leagues two or three” indicated in the sea-song “A joyful new ballad” (cited in \citealt{Palmer1986}: 13,) there was no significant evidence of this in the \isi{corpus}.} . Cardinal numbers were necessary in pre-nominal positions to mark plurality given the fact that bare nouns may have remained \isi{uninflected} and ordinal numbers were more likely in sequencing and specifically dates. Ordinal numbers used to express dates are commonplace in the \isi{corpus}, e.g., “the firstt of July” [DDB6 8/4] and “the fifteenth day of July last past” [HCA 1/9/139], “tenth day of Novr last past” [D/Earle/1/2] “the 23rd day of May” [HCA 1/13/97], and “the 10th of February” [HCA 1/99/3/10]. And although the prefixes <st>, <nd>, and <rd> are not always used consistently on numbers ending in either 1, 2, or 3, the sense of denoting ordinal sequence is clear, e.g., “The 22th of July” [HCA 1/9/8] and “the 2d of april last” [HCA 1/98/123]\footnote{The variation in ordinal suffix and orthographic representation use may be partially explained by the fact that the words “first” and “second” are not true ordinals, according to the \citet{oed1989} but rather nominal forms meaning “earliest” and “next” respectively, and as such, they were typically subject to greater variation than the true ordinals which are based on the cardinal forms, e.g., “Three” and “Third.”}. However, many dates expressed in the written records do not carry any type of ordinal marker, e.g., “8 October last” [HCA 1/99/87], “The 21 of October last” [HCA 1/14/140], “The 3 day of May” [DDB6 8/4] and “the 29 of Aprill last” [CO 5/1411/60]. This feature is mirrored in other references aside from expressing dates but where an ordinal number would be anticipated, e.g., “he was 2 \isi{Mate}” [HCA 1/99/59], “the 7 day we gott up” [DDB6 8/4], and “this 4 day” [DDB6 8/4]. However, given that these documents were composed before the standardization of English orthography and its imposition in public education coupled with the fact that many of the authors may have been only partially literate (see §3.11,) the use of cardinal numerals, e.g., “2” may have denoted both the word for the cardinal and ordinal number, i.e., “second” and “two.”\footnote{Although standard spellings for printed text were established by the end of the \isi{seventeenth century}, there was a lag in handwritten work as literate individuals did not necessarily reproduce standard forms in their private communication (\citealt{MillwardHayes2012}: 275). Idiomatic terminology may also account for examples like “Henry Every who was \textit{before} mate” [HCA 1/53/12], presumably meaning “first mate” given the context of the excerpt.}  This interpretation is supported by the fact that many of these numerals are preceded by either a \isi{definite article}, e.g., “the 7 day we gott up” [DDB6 8/4] and “the 29 of Aprill last” [CO 5/1411/60], or a \isi{demonstrative}, e.g.,~“this 4 day” [DDB6 8/4] suggesting that they should be spoken as ordinal and not cardinal numbers. Indeed, the apparent use of an ordinal \isi{noun} for a date followed by a cardinal numeral for a subsequent date in the following example “the firstt of July and the 3 of July” [DDB6 8/4] seems to owe more to orthography than it was likely intended to represent in speech. Yet, having acknowledged the likelihood of misinterpretation of this feature, it is nonetheless evident that some cardinal numbers were used to express dates and common expressions in sailors’ writing where we might anticipate ordinal determiners. ~

\subsection{{Sequence marking}}\label{sec:5.3.3}

Sequential ordinals such as “next” and “last” are a salient feature of sailors’ speech in the \isi{corpus} and this is likely because deponents were required to specify the dates and sequences of events to recreate projected timelines leading up to an alleged crime. The most prominent sequential marker used in \isi{noun} phrases was “last,” which was used in 118 examples or over two thirds of the sequential determining phrases sampled from the \isi{corpus}, followed by the idiomatic phrase “last past” which was used in 28 examples or one sixth of the 171 examples collected (see \tabref{tab:key:5.1}). Less frequent examples of ordinal and temporal markers included the words “past,” “since,” and “following” none of which composed more than one twentieth of all the samples collected. The most common marker “last” was used most commonly to refer to periods of time such as months, e.g., “in July last” [HCA 1/99/155] and “a little before Christmas last” [HCA 1/9/39]. The second most common referents were specific dates, e.g., “the 20th of Aprill last” [ASSI 45/4/1/135/10] and “the 1st Day of October last” [HCA 1/99/170]; and the least frequent type of referent was a day that was not described with a numerical date but was nonetheless specific, e.g., “on Saturday last” [ADM 52/1/8] and “Ten days before Christmas last” [HCA 1/14/54]. This scale of frequency may relate to the quantity of testimonial material in the \isi{corpus}, yet given that letters, logbooks, and miscellaneous documents also reflected this trend in ordinal marking with specific types of referent, it may reflect a wider feature of \isi{language use}. Further support shows in the fact that the markers “last past” and “past” are also most commonly used with periods of time, e.g., “In the month of June last past” [HCA 1/9/63], “at the beginning of March last past” [HCA 1/52/100], and “Febuary Past” [HCA 1/98/66], and “Four yeares past” [HCA 1/14/38]. Significantly fewer examples were found of the temporal marker “last past” used with specific days but without dates, e.g., “Upon tuesday or wednesday last past” [HCA 1/10/9] and “seaven night last past” [HCA 1/9/63]; and there were no examples of the marker “past” used with specific days but without dates, instead all examples were either periods of time or dates. The temporal marker “following” was identified only nine times in the sample material and there was no visible preference for what type of referent was used. The temporal marker “since” was the least used of all the temporal markers studied, yet showed an interesting trend in that it was used exclusively with periods of time, e.g., “about half a year since” [HCA 1/13/98] and “about six months since he was first taken” [CO 5/1411/97] comparable to the usage of the word “ago” in modern speech. Thus, overall, the data show that when using sequential ordinals in time references, “last” is the favored marker and among all variants and periods of time are favored over specific day or date references. 

\begin{table}
\caption{\label{tab:key:5.1}: Frequency and type of sequential ordinals with types of referent in 171 examples\\
{\tiny Sampled from collections 1045.f.3, 445f.1, ADM 106/288, ASSI 45/4/1/135, CO 5/1411, HCA 1/9-14, D/Earle/1/2, HCA 1/52-53, HCA 1/98-99, HCA 1/101, SP 42/6, SP 89/25, \& T/70/1216.}
}
\footnotesize
\begin{tabularx}{\textwidth}{l@{\qquad}rrrrrY}
\lsptoprule
 & \multicolumn{5}{p{7.5cm}}{\centering\hspace*{2mm}\textbf{Frequency by type of sequential marker} \newline \centering\noindent\hspace*{2mm}Number of uses (percentage of examples per marker)} & \\
\textbf{Type of referent}& `Last' & `Last past' & `Following' & `Past' & `Since' & \textbf{Total}\\
\midrule
Period of time & 50 (42\%) & 14 (50\%) & 3 (33\%) & 7 (77\%) & 7 (100\%) & \textbf{81}\\
Specific date & 43 (36\%) & 11 (39\%) & 3 (33\%) & 2 (22\%) & 0 & \textbf{59}\\
Specific day & 25 (21\%) & 3 (11\%) & 3 (33\%) & 0 & 0 & \textbf{31}\\
\midrule
 \textbf{Total} & \textbf{118}    &  \textbf{28} &     \textbf{9} &  \textbf{9} &  \textbf{7} & \textbf{171}\\
\lspbottomrule
\end{tabularx}
\end{table}

Sequential markers show trends of usage in relation to estimation with certain markers being used exclusively with estimated times and others seemingly not permissible with estimation. The marker “since” is used exclusively with estimated periods of time in the seven samples of usage recorded in the sample, e.g., “about 3 weeks or a month since” [HCA 1/13/97], “he arrived about two months since” [HCA 1/13/97], and “about three years Since” [HCA 1/14/20]. Comparatively, none of the 18 markers using either “following” or “past” were used with an estimated time period or date suggesting that the combination is either not permitted or unusual in Ship English. The two most commonly sampled markers, “last” and “last past” however, permit estimation, e.g., “about the middle of January last” [HCA 1/99/58], “either on Friday of Thursday last” [ADM 106/288/26], “about the 22nd of June last past” [HCA 1/10/2], and “In or about the month of June last past” [HCA 1/9/138]. Yet estimation was evident in only a tenth of the structures using “last” (12 examples of 118) compared to a fifth of the structures using “last past” (5 examples of 28,) so although the numbers of “last past” are fewer overall, the frequency with which they permit estimation is significantly higher than the marker “last” suggesting a general trend. 

The placement of sequential markers in predominantly \isi{post-nominal position} was also a feature of their usage. The majority of all the 171 examples taken to study sequential marking, and all of the markers using “last past,” “following, “past” and “since” were in \isi{post-nominal position}, e.g., “The 9th day of Aprill last past” [HCA 1/14/140], “we came thence latterly the 5th following” [SP 42/6], “this 3 weeks past” [ADM 106/300/52], and “about five months since,” [HCA 1/13/97]. The most commonly used marker analysed in \tabref{tab:key:5.1}, the word “last,” although it was also used overwhelmingly in post-nominal positions, also permitted pre-nominal usage, e.g., “at 4 Last night” [ADM 52/1/8], “on the last day of September” [HCA 1/98/255], “Last Christmas” [HCA 1/13/96], and “last April” [HCA 1/99/130]. Although pre-nominal examples were few in comparison to the number of post-nominal examples (only 10 examples or 8\% of the total 118,) the finding is significant in comparison with the complete lack of pre-nominal usage for other markers. Additionally, less-used markers such as “next,” were permitted in pre- and post-nominal positions, e.g., pre-nominal “went thither again the next morning” [HCA 1/14/51] and “the next day” [HCA 1/14/201] compared to post-nominal “before March next” [HCA 1/14/20] and “on Wednesday next” [HCA 1/99 \textit{The American: Weekly Mercury} No.618, Oct 28-Nov 4 1731]. The fact that certain words such as “last” and “next” were permitted in pre-nominal and post-nominal positions suggests that either these were the first sequential markers that were showing movement into the determiner position from a more common default post-nominal modifying position as part of a wider process of \isi{language change}, or that such variation was potentially conditioned and specific to Ship English. 

\subsection{{Quantifying mass nouns}}\label{sec:5.3.4}

Mass nouns, specifically referring to the weather, were common in Ship English. Climatological phenomena such as rain, wind, and lightning are often referred to with the mass determiner “much” or “some” to denote quantity, e.g., “the weather cold the last night some Raine” [ADM 52/2/5], “fair weather for the most part som raine” [ADM 52/2/8], “very much Rain” [ADM 52/1/7], "with mutch littning and Raine" [DDB6 8/4], “we had much winds” [ADM 52/1/4], and “wee had mutch Westterly winds” [DDB6 8/4]. The word “gust” referring to a localized strong current of wind although commonly treated as a \isi{count noun} in modern English appears to have also been used as a \isi{mass noun} in Ship English, e.g., “blowing hard in gust” [ADM 52/1/7] and “with Gust and Rain” [ADM 52/1/7], although it may be that the use of the \isi{uninflected} form in the last example might have been conditioned by correlation with the \isi{mass noun} “rain” as there are also examples of how the supposedly \isi{mass noun} “rain” is inflected for \isi{plural} marking in correlation with the inflected form “gust,” e.g., “very hard gales with much Raines” [ADM 52/1/1]. Hence, it might be that the grammatical correlation that requires the same nominal constituents (i.e., inflected \isi{noun} + inflected \isi{noun}) takes precedence over inflectional morphology at the lexical level based on whether the \isi{noun} is a \isi{mass noun} or a \isi{count noun}. 

Abstract nouns commonly expressed in mass form could be realized as singular count nouns and with \isi{plural inflection} in Ship English. For example, the word “evidence” is realized as a singular \isi{count noun} determined by the singular \isi{demonstrative} “this” in the example “this evidence that have been already produced” [CO 5/1411/33], and the \isi{noun} is explicitly inflected for \isi{plural} marking in conjunction with a \isi{cardinal number} in the example “three evidences” [HCA 1/9/51]. The \isi{plural inflection} of this word is similarly realized alongside \isi{indefinite} pre-articles, e.g., “more evidences wou’d have appeared” [HCA 1/99/36], “produced Several evidences in their Behalf” [HCA 1/99/69], and “I have a great many evidences” [CO 5/1411/41]. Other abstract and mass nouns common to court proceedings such as “advice” and “information” are similarly expressed as count nouns, e.g., “The Advices Your Honour is pleased to favour us with” [HCA 1/99 \textit{The American: Weekly Mercury} No.617, Oct 21-Oct 28 1731], “an information” [HCA 1/14/150], and “his only informations” [HCA 1/99/42]. Additionally, other common mass nouns such as “work” and “money” are realized as singular count nouns or inflected plurals in the \isi{corpus}, e.g., “this provision must be a work of some time” [BL/Egerton 2395/0007], “takeing out moneys” [ADM 52/3/7], and “confes, where their moneys was” [HCA 1/9/18]. Such evidence appears to suggest that categories of \isi{mass noun} and \isi{count noun} were not mutually exclusive and that certain lexical items could be realized as \isi{uninflected} mass nouns in addition to being used as count nouns with \isi{plural inflection} and singular-referent articles and pre-articles. 

Determiners specific to mass nouns such as “much” “little” and “small” are frequent in Ship English. The use of the determiner “much” in correlation with mass nouns in both logbooks and witness depositions appears to be common in the context of \isi{indicative modality}, e.g., “Much Thunder lightning and Raine” [ADM 52/1/1], and “with much Rain” [ADM 52/1/7], and “he saw much blood” [HCA 1/101/405]. And this usage is particularly pronounced in reference to the wind, e.g., “much wind all the night” [ADM 52/3/12], “very mutch Raine and wind” [DDB68/4], and “much wind” [ADM 52/1/10, ADM 52/2/3], although other abstract nouns and adjectives such as “trouble,” “afraid” and “out sailed” show similar usage of the determiner “much” in the context of \isi{indicative modality}, e.g., “they had much trouble to gett [a pilot]” [ADM 52/2/9], “but were much afraid of us” [T/70/1216/9], and “the Dutch frigatt who we Much out Sailed” [ADM 52/2/9]. This usage in the context of \isi{indicative modality} appears in contrast to the common native usage of the word when it is conditioned by \isi{interrogative modality} or negation in phrases such as “Is there much wind?” or “there wasn’t much wind.” Other examples such as the use of “any” in \isi{indicative modality} in the example “they had sold any timber” [HCA 1/9/53] - meaning that they had sold \textit{some} timber - suggests that constraints assigning determiners to specific modalities was not as strict as we might anticipate in contemporary standard varieties of English.  

The determiner “small” is used with \isi{uninflected} mass nouns to denote a small amount, e.g., “small drizzling rain” [ADM 52/2/3] and~“some small Raine” [ADM 52/2/5] in addition to its use as an \isi{adjective} with count nouns, e.g., “small Arms” [HCA 1/98/271] and “a Small Hoy” [ADM 52/2/3]. This determiner is also commonly used with \isi{uninflected} mass nouns to denote a small amount, e.g., “little or noe news” [ADM 52/1/8] and “little sugar to be had” [T 70/1/9]; yet examples show that the determiner “little” is almost universally used with the nominal “wind,” e.g., ~“blowing unconstant sometime little wind” [ADM 52/1/7], “Little wind and much lightning” [ADM 51/3946/6], “foggy at the forenoon at afternoon little wind” [ADM 52/2/2], “feare weathr but Little wind” [T/70/1216/10], and “Little wind” [ADM 52/3/12]. However, the predominant use of the determiner “little” with the \isi{noun} “wind” does not condition the \isi{noun} to appear in an \isi{uninflected} form as a \isi{mass noun} but also permits inflection and use of the nominal as a \isi{count noun}, e.g., “little winds” [ADM 52/2/5], “littel winds” [HCA 1/99 Log Book Pideaux 1731], “and little winds” [DDB6 8/4]. And this usage seemingly extends to other typically mass nouns, e.g., “our monys being little and not enough” [T 70/1/10]. Therefore, we may surmise that although the determiners “much” “little” and “small” are commonplace in Ship English, they each have specific constraints reflating to modality, use with mass and count nouns, and lexical projection that do not seem to be universal among the determiner class.  

\subsection{{Articles}}\label{sec:5.3.5}

Variation in the use of \isi{indefinite} articles does not feature heavily in the \isi{corpus}, but examples of their usage may suggest some features of phonological realization and the customs of marking for \isi{indefinite} and specific referents. One example of the \isi{indefinite} article “a” used in a prevocalic position in the phrase “he hath been a Eye witness to many Moors Shipes” [HCA 1/98/24] may suggest that in certain contexts, the pronunciation of the article was realized as a vocalic attachment to a prior closed syllable. To illustrate, the excerpt “been a Eye” from the previous citation may have been realized phonetically as /b{ɪ}nʌ{aɪ/ in which the article “a” is realized as an unstressed vocalic nucleus (specifically, a caret) attached to a second syllable that re-assigns the final coda /n/ of the syllable “been” as the onset of the newly created syllable that the article attaches to. The syllable division, in addition to the unstressed caret would help avoid potential cacophony with the rising diphthong at the onset of the word “eye.” However, such conjecture is nonetheless dependent on specific preferences for vocalic realization that are extremely difficult to determine from written samples. Another feature of \isi{indefinite} article usage suggested by examples in the \isi{corpus} is that they were permissible in \isi{pre-nominal position} with generic abstract referents, e.g., the abstract nouns “courage,” and “prey” and the abstract \isi{adjective} “french” in the examples} “the capt of the Pyrates bid me have a good courage” [CO 5/1411/35], “those men should become a Prey” [BL/74/816/m/11/36/1], and “Being a french men” [HCA 1/14/38]. Furthermore, in addition to \isi{indefinite} articles being {permissible for generic abstract referents, they also appear to have been acceptable for specific singular referents (more commonly denoted with the \isi{definite article}], e.g.,} “Ships Company espying in a morning Severall pasengr boates” [HCA 1/52/88] and “he looked upon a chart and shewd us way they were to go” [HCA 1/99 The Tryals of Agostinho, July 8, c.1721, 7]. In both previous citations, the specific “morning” and the specific “chart” might be anticipated to have been referred to using the \isi{definite article} “the” prior to the \isi{noun}. Thus, although examples of such usage are not extensive, they certainly suggest accepted (if potentially localized) variation in usage. 

Omission of articles, both \isi{indefinite} and definite, is a much more prevalent feature in the \isi{corpus} than variant usage. The following four examples taken from a witness disposition, two letters, and a journal entry omit the \isi{indefinite} article: “he was [a] very good man” [D/Earle/3/1], “about [a] fortnight before” [HCA 1/101/46], “Within [a] few days” [BL/Egerton 2395/0003], and “they have [a] variety of forecastle songs” (cited in \citealt{Palmer1986}: 104). Other examples omit an article that could have been realized as an \isi{indefinite} or a \isi{definite article} depending on context: “[a/the] great quantity of goods” [HCA 1/53/12], “I could not have [a/the] opportunity to speak with him” [445f.1/27], “This morn: had [an/the] order to go for Plimouth” [ADM 52/1/1], “gave him [an/the] account,” and “on [a/the] promise that” [HCA 1/99 \isi{Bahama} Islands 1722]. Other examples of omission are clearly referring to specific vessels or parts of the ship that would typically select a \isi{definite article}, e.g., “his majesteys ship [the] \textit{\isi{Essex} prize}” [CO 5/1411/653], “last night at 12 umoored [the] ship” [ADM 52/1/5], “command was then given to shorten [the] saile” [HCA 1/9/155], “winds...last night struck [the] yards \& topmasts” [ADM 52/2/3], and “we gett [the] Anchor aboard” [ADM 52/3/7]. Such usage is mirrored in specific references to ranks and concrete nouns, e.g., “he was [the] Boatswaine[s] \isi{Mate} of an English ship” [HCA 1/53/10] “he was absent from [the] house” [HCA 1/101/425], and “I went...for to selle at [the] ffartory” [T/70/1216/13]. It may have been that the abbreviated nature of ships logbooks and court documents rendered articles unnecessary, as we might infer from the lack of a \isi{definite article} to refer to the “\isi{prisoner}” in the example “whether he ever saw [the] said \isi{prisoner}” [HCA 1/99 New Providence 1722]. However multiple examples of ships logbooks and Admiralty court documents that make use of the \isi{definite article} seem to contradict the suggestion that the omission was associated with the accepted styles of the written modes, e.g., “after they of the \textit{Briganteen} had plundered the said ship the \textit{Sea Flower} and taken out of her what they thought fitt they coming again about the sd \textit{Briganteen} put the mast of the \textit{Sea flower”} [HCA 1/53/57]. This example shows \isi{definite article} usage in all of the phrases that refer to both ships and the equipment of the ship, and even uses the \isi{definite article} in the expressions “the said ship” and “the sd \textit{Briganteen}” that are explicitly marked as formal courtroom utterances. Another possibility to explain the omission of articles is that the expressions in which articles are omitted are idiomatic, e.g., the expression “made [an/the] oath” [CO 5/1411/640] in a letter circa 1697 that is repeated almost exactly in many other unrelated documents e.g., “made [an/the] Ooath” in a \isi{witness deposition} in 1700 [SP 42/6]. Indeed, whether it was because of \isi{idiomatic usage} or characteristic variation, the omission of articles is salient in a range of documentation that includes witness testimony, personal letters, logbooks and journal entries written by or on behalf of mariners. 

It appears that rather than something that was conditioned by linguistic constraints, the omission of articles was a \isi{free variation} for many speakers of Ship English. This assertion is based on a number of documents written in the same hand, and so therefore assumed to be by the same person, showing variation between the use of articles and their omission with similar nominal forms in comparable structures and in a single \isi{speech act}, e.g., the author of one logbook uses the \isi{definite article} for “the wind” but then omits the \isi{definite article} prior to the cardinal direction “west” in the entry “att 10 att nightt the wind came to [the] west” [DDB6 8/4], and the author of another logbook uses multiple definite articles (both orthographically represented as “ye” and “the”) for weather and compass point directions in addition to referring to the ship’s cargo hold, but then omits the anticipated \isi{definite article} when referring to the ship’s decks in the entry “This 24 houres ye wind from the WS to the SO and back to ye SWbW Stored downe into ye hould betwixt [the] decks” [ADM 52/2/3]. In another example, a logbook entry includes the excerpts “the \isi{long boat} came a boord with provisions…[and] yesterday afternoon [the] \isi{long boat} went to pagan Crook for provisions” [CO 5/1411/712] using the \isi{definite article} for “\isi{long boat}” in the first clause but omitting it in the second. Two pages later, this same author uses the same syntactic construction as the second clause which omitted the \isi{definite article}, but this time the \isi{definite article} is present, “yesterday afternoon the \isi{long boat} came from york” [CO 5/1411/714]. Such \isi{free variation} was also a common variant in court documentation, e.g., the omission and then use of the \isi{definite article} prior to the \isi{noun} “\isi{prisoner}(s)” in the example, “[the] \isi{prisoner} having nothing more to say. The Prisoners were ordered to withdraw” [HCA 1/99 New Providence 1722]. Although it is possible that logbook authors might have purposefully oriented themselves towards formal court language in logbooks, it may also have been that clerks of the High Court of the Admiralty acquired variant maritime usage from exposure to multiple examples in sailors’ speech through their job requirement of having to write sailors’ depositions. 

The \isi{definite article} was permitted with specific semantic fields and appears to have been conditioned by adverbial \isi{gerund} phrases and to avoid null categories in the determiner position. Gerund phrases may take a \isi{definite article} in Ship English, specifically when used in adverbial constructions, e.g., “order to \textit{the having} of her secured” [CO 5/1411/653 emphasis added]\footnote{Italic font emphasis is added to all examples of \isi{noun} phrases with definite articles in this paragraph} and “by \textit{the not keeping} their apparel sweet and dry, and \textit{the not cleansing} and keeping their cabins sweet” (cited in \citealt{Brown2011}: 64). However, even in constructions not subordinated by adverbial markers, gerunds may be preceded by definite articles, e.g., “meaning as this Deponent understood, \textit{the Running-away} with the Pink” [HCA 1/99 \textit{The American: Weekly Mercury} No.618, Oct 28-Nov 4 1731]. In addition to using definite articles with \isi{gerund} phrases, Ship English also permits the use of definite articles with newly-introduced and generic referents, e.g., one \isi{sailor}’s explanation of how wooden ships are damaged because “\textit{the worm} comes in… [hulls] being much dammaged by \textit{the worm}” [CO 5/1411/651]. Following \citeapo{Hawkins1978} classification of \isi{definite article} usage in English, unless the reference to “worms” was anaphoric, associative, or indexed some kind of previously-established shared knowledge, then the use of the \isi{definite article} \isi{noun phrase} “the worms” rather than the generic \isi{plural} “worms” is atypical. Yet, the use of definite articles in such contexts is common, particularly with abstract nominals, e.g., “the Discouragement which these proceedings bring to \textit{the Navigation}” [BL/74/816/m/11/36/1], “of \textit{the Cash} he hath not one penny” [HCA 1/98/259], “taking \textit{the command} over them” [HCA 1/99 Cape Coast of Africa, Feb 4 1734, 4], and “not being perfectly versed in \textit{the English}” [HCA 1/99 The Tryals of Agostinho, July 8, c.1721, 7]. Furthermore, temporal references permit the use of definite articles, specifically in logbook entries, e.g., “The wind was moderate all \textit{the morning}” [ADM 51/3797/1], “the weather cold \textit{the last night} some Raine” [ADM 52/2/5], “Little wind all \textit{the night}” [ADM 52/2/5], and “with a fresh gale of wind all \textit{the 24 hours}” [ADM 52/2/5]. It may be that time references in logbooks required an explicit determiner and if no other determiner was used, the \isi{definite article} served to avoid a null constituent\footnote{The suggestion that Ship English preferred multiple determiners over null constituents in determiner position is further supported using demonstratives alongside \isi{possessive} pronominal determiners, e.g., the appearance of “this his” in the testimony “sufficient force to defend this his colony \& dominion” [CO 5/1411/630] and competing determiners, e.g., “this last 24 hours wee had the wind variable between the NNWt and NEt” [ADM 51/3983/1.]}. Yet, the \isi{definite article} was also used in ways common to Early Modern English such as referring to locations, e.g., “being bound for \textit{the Barbados}” [T/70/1215 29th Oct] and “designing for \textit{the Havana}” [HCA 1/99 \isi{Bahama} Islands 1722] and in certain idiomatic expressions, e.g., “the Master would not agree to this, but kept \textit{the Sea}” [445f.1/33]. Thus, although definite articles followed many of the accepted usage constraints that we might see in other varieties of Early Modern English, it appears that Ship English also permits the \isi{definite article} with specific abstract and temporal semantic fields and in the linguistic contexts of \isi{gerund} phrases and otherwise null determiner categories. 

\section{{Pronouns}}\label{sec:5.4}

\subsection{{Heavy use of pronominal forms}}\label{sec:5.4.1}

The \isi{corpus} shows heavy use of pronominal forms including nominative, \isi{accusative} and \isi{genitive case} variants in a single phrase or clause. Court depositions show particularly heavy personal \isi{pronoun} usage in short excerpts, e.g., “he was sorry he was sick, and could not goe with him, he taking a Laced Hat away” [HCA 1/99/38] including three instances of the singular male \isi{third person} nominative form and one instance of the \isi{accusative} form; “Capt Fairbourne, to whom I delivered the letter you sent me for him…presented a salute from him” [SP 42/6] including two instances of the nominative form and three instances of the \isi{accusative} form; and “the Quarter Master called him out of the Boat Several times, which he not obeying presently, beat him severely for, in his sight” [HCA 1/99/166] including one instance of the nominative form, two instances of the \isi{accusative} form, and one instance of the \isi{genitive} form\footnote{Although case marking is discussed under the assumption that classifications follow the same parameters as \isi{modern usage}, there is some data to suggest that variation was permitted or the constraints which selected case marking were not as fixed as we might anticipate, e.g., one logbook entry reads, “meridian morinlgo beard from we,” [HCA 1/99/29]. The verbal marker “beard,” most likely to be a past form of the nautical term “to bear” with a regularized inflection, projects a \isi{prepositional phrase} “from we” expressed with a nominative case \isi{pronoun} “we” and not the anticipated \isi{accusative case} “us.”} . Furthermore, many pronouns are redundant in their linguistic contexts, such as pronouns used to refer to human subjects who are also named in an adjacent \isi{noun phrase}, e.g., “Mr Harley he’s got 203 cheeses” [HCA 1/101/541], “him who is Your loving Fried Fra. Nicholson” [CO 5/1411/652], “him the Deponent” [HCA 1/99/3, HCA 1/99/5, 1/99/7], and “He(e) this \isi{deponent}” [HCA 1/9/10, HCA 1/14/17]. Prenominal redundancy also features in the idiomatic expression “on board of…” or “a board on…” followed by an \isi{accusative} \isi{pronoun} (typically “him” but also potentially “them” and “us”) and was repeated numerous times in the \isi{corpus}, e.g., in HCA 1/98/182, CO 5/1411/98, CO 5/1411/99, 445f.1/34, ADM 52/1/7, ADM 52/1/8, and HCA 1/99/80. This idiomatic structure makes use of a \isi{prepositional phrase} with an \isi{accusative} object of the \isi{preposition} yet the structure is redundant in most cases as the vessel boarded is made clear in context e.g., “fall with the said fisher boat and remained them a board on him” [HCA 1/101/431] and “this same Moody was one of the Pyrates, that came on Board of him in the Boat” [HCA 1/99/37]. Interestingly, the last example also makes use of a double marked nominal “Moody…one of the pyrates” in addition to the double marked object “on Board of him… the Boat” further suggesting the nature of heavily marked nominal forms in Ship English. Redundant prepositional phrases of agency including pronominal objects of the \isi{preposition} and grammatically unnecessary markers of the \isi{indefinite} object also feature in the \isi{corpus}, e.g., “was Burnt by us” [ADM 52/1/1], “his own confession to him” [HCA 1/99/22], and “I can demonstrate to you” [BL/74/816/m/11/36/2]. In short, although sailors’ speech was characterized by heavy pronominal use, many instances of personal pronouns are redundant in context suggesting that they were customarily used in acts of over-specification.  

\subsection{{Possessive pronouns}}\label{sec:5.4.2}

In addition to the \isi{possessive} pronouns that mark \isi{genitive case} marking (also discussed in §5.2.2,) pronominal forms that function as determiners in \isi{noun} phrases, e.g., “his” and “your,” commonly occur in collocation with \isi{gerund} phrases. One \isi{sailor}’s letter shows this structure in a \isi{prepositional phrase} of reason, “thanke you for \textit{your sending} us the Navy-Yacht” [ADM 106/288/30 emphasis added]. In this specific example as well as others, the \isi{possessive} pronominal determiner might have been omitted without affecting meaning and this suggests that when gerunds are used in Ship English, they involve linguistic constraints that select determiners. Furthermore, these \isi{pronoun} and \isi{gerund} constructions often function in lieu of subordinating clauses, e.g., one \isi{witness deposition} includes the phrase “\textit{his getting} a certificate clandestinly sign'd by sevll masters of ships in this land” [SP 42/6 emphasis added] that opens with a prenominal \isi{possessive} determiner “his” and forms a \isi{noun phrase} with the \isi{gerund} “getting a certificate” that is itself modified by the phrase “clandestinly sign'd by sevll masters of ships in this land.” This construction forms a lengthy \isi{noun phrase} that might have been expressed as a subordinated clause, “[because] he got a certificate clandestinly sign'd by sevll masters of ships in this land.” Other examples with emphasis added to the constructions in question are “[he] knows nothing of \textit{his being} ashoare” [HCA 1/99/8], and “[I] don’t Remember \textit{his going} Particularly on Board” [HCA 1/99/91]. In both these examples, the pronominal determiners “his” preceding the subsequent \isi{gerund} forms “being” and “going” avoid the need for a secondary \isi{relative clause} and instead embed the information in the \isi{primary clause}. Both examples might be expressed as compound sentences using the \isi{relative pronoun} “that,” e.g., “he didn’t know \textit{that} he was ashore” (for [he] knows nothing of \textit{his being} ashoare], and “I don’t remember \textit{that} he went on board particularly” (for [I] don’t Remember \textit{his going} Particularly on Board). Yet, speakers and writers of Ship English consistently show an avoidance of such relative clauses and instead prefer to build complex matrix clauses with multiple embedded \isi{noun} phrases\footnote{Millward and Hayes explain that Early Modern English favored long and heavily subordinated constructions that emulated Latinate style that built upon older native traditions of cumulative, paratactic sentences that were never completely lost (2012: 188: 275-276). It may be that the complex matrix clauses with multiple embedded \isi{noun} phrases of Ship English derive from the Old English preference for cumulative, run-on clauses which were reinforced by more recent preferences for \isi{subordination}.} . Although further research would be needed to confirm these constraints, the examples presented here suggest that determiners were not only acceptable but also potentially necessary in \isi{gerund} phrases expressing attendant circumstances and that this contributed to a wider phenomenon of compounding complex matrix clauses rather than using subordinating clause structures.

Possessive pronouns such as “yours,” “mine,” and “hers” are not frequently used in the \isi{corpus}, but do occur in some personal communications. One letter begins “yours of the third I received on Saturday” [CO 5/1411/647] and because this is the opening line of the letter, the \isi{possessive} \isi{pronoun} “yours” has no antecedent and so the reader can only assume the author meant “your letter.” This use of \isi{possessive} pronouns without a prior antecedent is repeated in other letters among sailors, e.g., “I received yours of the Second of october 1693” [HCA 1/98/56] and “hopes may hear from you in answer to ours” [T 70/1/12]. This type of usage may have been a dialectal or idiomatic variant specific to personal communication in Early Modern English among the literate classes, but it does not appear in either of the two widely circulated epistolary narratives of the seventeenth, specifically James Howell’s \textit{Familiar letters} (1645-1650,) Aphra Behn’s \textit{Love-letters between a nobleman and his sister} (1684-1687). However, this use of \isi{possessive} pronouns without a specific anterior referent does make an appearance in \isi{eighteenth century} fiction, specifically Samuel Richardson’s widely popular epistolary novel \textit{Pamela…} (1740,) e.g., “we had not read through all yours” and “I see yours is big with some important meaning” (both examples from Letter XXXII). Yet the usage is not frequent and does not appear at all in the same author’s nine-volume \textit{Clarissa Harlowe…} written only nine years later, suggesting that although the feature was an acceptable variant in contemporary English, it was certainly not common, even among the specific groups of people who used it. Literate sailors who used \isi{possessive} pronominal forms without specific anterior nominal markers may therefore be one of the small groups for whom this usage was acceptable as early as the \isi{seventeenth century}. 

\subsection{{Expletives}}\label{sec:5.4.3}

Pronominal expletives are commonly used with references to the weather but can be omitted in reference to intangible referents.  Logbook entries often include references to the weather using the \isi{expletive} marker “it” (indicated with italic emphasis,) e.g., “This 24 hower we have had \textit{it} for the moss part Calme” [ADM 52/2/10]. “\textit{it} began to raine to raine have been little wind” [ADM 52/2/3], and “this morn \textit{it} came to WNW a very hard Gale” [ADM 52/1/1]. This is even more common with references to wind conditions, e.g., “\textit{It} blew a hard gale” [ADM 52/3/12], “\textit{It} continued a fresh Gaile” [ADM 51/3797/1], and “\textit{It} was soe much wind...wee were forced to lay under our Maine Sayle” [ADM 51/4322/1]. References to storm conditions use similar constructions, e.g., “All last night \textit{it} blew a Storme” [ADM 51/3954] and “last night \textit{it} began to blow and this morning it encreasing to a Storme” [ADM 51/3954]. Yet, in contexts with no tangible referent such as wind, rain, or a storm, expletives are infrequently used or omitted completely, e.g., “I suppose [it] was in regard [of] so many” [CO 5/1411/41] and “in March last [it] was two years he sailed out of the River of Thames” [HCA 1/14/201]. Thus, we might surmise that although \isi{expletive} pronouns are evident in the \isi{corpus}, their use was localized, or at least preferred, with tangible referents rather than being used to function as the grammatical subjects of existential constructions.

\subsection{{Indefinite pronouns}}\label{sec:5.4.4}

Indefinite pronouns are not common in either witness depositions, logbook entries or personal communications, instead, sailors used adjectives to indicate generic referents. References to a sizable but \isi{indefinite} group of people (most commonly an entire or partial \isi{crew}) often include the particle “all” to index large but \isi{indefinite} numbers, e.g., “Ye have all of you been wickedly united” [HCA 1/99/3/2] and “we thought all that you had been sick” [HCA 1/101/541]. The phrases “all of you” in the first citation and “we…all” in the second might be understood as “everybody” or “everyone,” yet instead of using the \isi{indefinite} pronouns that were fairly recent developments in Early Modern English, sailors seemingly preferred to use personal pronouns such as “you” and “we” respectively alongside a modifying particle. This same strategy is evident with the use of the term “every” that does not always compound to form an \isi{indefinite} \isi{pronoun} but can function as a modifying \isi{adjective} particle with subject pronouns, e.g., “requested that you and every [one] of you” [HCA 1/98/53]. The assumption that the nominative form “one” exists in the underlying structure of the previous example is based on other examples of usage, e.g., “every one that take it will make use of their time” [ADM 106/288/31] and “every one came” [HCA 1/99/59], in which the phrase “every one” is not conceptualized as a single \isi{pronoun} but rather a \isi{noun phrase} composed of “one” modified by the \isi{adjective} “every”\footnote{Note that this is in contrast to the \isi{indefinite} pronouns “everything” and “somebody” that do appear written as one \isi{lexeme} in the \isi{corpus}, e.g., “hacking everything” [HCA 1/99/126] and “somebody had stole the boat” [HCA 1/99 \isi{Bahama} Islands 1722.]} . Yet, Millward and Hayes explain that in the Early Modern English period it was acceptable to use “every” as an independent \isi{pronoun} meaning “all” rather than solely as a pronominal \isi{adjective} (2012: 264) and the \citet{oed1989} states that the particle “every” was commonly used as an \isi{adjective} in collocation with nouns like “one” and “body” to create \isi{indefinite} \isi{noun} phrases up until 1820 with orthographically joined examples of the words not appearing until the \isi{eighteenth century}, e.g.,  Defoe’s use of “everybody” (cited in \textit{Oxford Eng. Dict.} 1989, Vol 5: 466). It seems that although conjoined forms were becoming acceptable in the period under study, sailors continued to mark \isi{indefinite} nouns using \isi{adjectival} particles and pronominal forms, or to use particles such as “all” and “every” as independent pronouns themselves.

Like constructions using “all” and “every,” phrases with explicit negation or \isi{interrogative modality} were also constructed with \isi{adjectival} particles that could function as independent pronouns. References to zero people (i.e., nobody) and zero items (i.e., nothing) were commonly denoted with the particle “none” functioning as an independent \isi{pronoun}, e.g., “if you maet none” [CO 5/1411/657], “could find none at all” [T/70/1213], “eaten by none else” [1045.f.3/1/28: 25], “could get none” [HCA 1/9/155], and “there being None in Stores” [ADM 106/288/32]. There are examples of the usage of the particle “any” as an independent \isi{pronominal form}, e.g., “nott any that wee did like” [T/70/1213], “if they did not meet with any to go” [HCA 1/99/6], and “if any such there be” [CO 5/1411/649]. Yet, the word “any” was more frequently used in \isi{sailor}’s speech as an \isi{adjectival} particle, most commonly with the word “person,” e.g., “he did noe hurt to any person” [HCA 1/9/67], “nor shall carry off this land any Persons” [HCA 1/9/7], and “nor with any other person” [HCA 1/9/63]. The use of the particle “any” with the \isi{nominal form} “one” was also evident, although whether this was considered in its underlying form as either a \isi{noun phrase} composed of an \isi{adjective} and \isi{noun}, or a pronominal \isi{lexeme} is unclear - and orthographic spacing seems to imply that both interpretations are valid, e.g., “I lay not this as an injunction upon any one” [445f.1/22] and “did no harm to any of them, nor never Intended to Kill anyone” [HCA 1/99/11]. The interpretation that the construction “any…one” was used as an \isi{indefinite} \isi{pronoun} is supported by similar and repeated constructions of the word “any…thing,” e.g., “neither he nor any of the others understand any thing of navigation” [HCA 1/99/11], “Did you heare him say any thing” [CO 5/1411/29], and “but did not before know anything of them” [HCA 1/13/92]. So, although the particles “none” and “any” are both used as independent pronouns in the same way as “all” and “every,” they were also used by sailors as \isi{indefinite} adjectives in apparently \isi{free variation}.

\subsection{{Reflexive pronouns}}\label{sec:5.4.5}

Reflexive pronouns in the \isi{corpus} are sometimes used in ways that align with common usage, yet they also have variation that was not common in the Early Middle English period. Sailors used reflexive pronouns typically when the object of the sentence is the same as the subject, e.g., “I…haveing no mony nor frinds to helpe my selfe” [HCA 1/12/36], “he thought himselfe in ill companie” [ASSI 45/4/1/135/4], “he would defend himselfe” [HCA 1/52/46], “wee doe not wrong our Selves” [HCA 1/9/13], and “seamen belonging to the sd ship \textit{\isi{Essex} Prize}, which allready have, or here after shall absent themselves from his majestys service” [CO 5/1411/667]. Yet, even though the use of reflexive pronouns had been established since Middle English in the \isi{object position} of a sentence that referred back to the \isi{nominal subject} (\citealt{MillwardHayes2012}: 263,) sailors commonly omitted anticipated reflexives and instead used \isi{accusative} personal pronouns, e.g., “John Wingfield Swears him[self] to have been a Loving Man” [HCA 1/99/38], “he said in defense of him[self]” [HCA 1/99/48], “he really believes him[self] to have been the instrument of saving her” [HCA 1/99/50], “he had confessed to him[self] a great deal of sorrow” [HCA 1/99/50], and “George Freeborne took upon him[self] to be a Man of War” [HCA 1/9/14]. It seems that this was a salient feature of sailors’ speech as it is represented in a popular sea-song of the early \isi{seventeenth century}, “John Dory bought him[self] and ambling nag” (cited in \citealt{Palmer1986}: 1). Yet more commonly than replacing the \isi{reflexive pronoun} with another \isi{pronominal form}, evidence from the \isi{corpus} shows that sailors used reflexives in contexts where they were not typical, such as to show \isi{genitive case}, e.g., “it is the opinion of my self” [CO 5/1411/647], and to refer back to a subject that was not the subject of the anterior clause, e.g., “he took in the water, and likewise my self again” [CO 5/1411/639], meaning that he (the accused) took caskets of water on board and also took the witness (i.e., “me”) on board again. However, the most common variant usage of reflexive pronouns in the \isi{corpus} of Ship English was in the subject position of a clause where a nominative (subject) form of the \isi{pronoun} would be more typical e.g., “to day my self made... survayed 5 butts in the hold” [ADM 52/1/6], “himself was beat, and forced from a good Employ” [HCA 1/99/80], “himself came to us” [T 70/1/10], and “Yesterday My Self with the Rest of the Foresignts Company were turned over” [ADM 51/4170/2]. Although not reproduced in full here, these examples were taken from larger legible utterances in which it was clear that the subject of the previous clause was not the same as the referent indicated by the \isi{reflexive pronoun}, as in the example “the \isi{prisoner} should go when himself [i.e., the witness] did” [HCA 1/99/48]. So, although reflexives sometimes followed common usage patterns in sailors’ communities, they were not semantically restricted to refer to the same subject referent of the clause in which they appear, nor were they grammatically restricted to the \isi{object position} of the clause. 

Sailors routinely used reflexives after certain verbs of personal expression. Specifically, verbs with a semantic link to oral expression often took a reflexive \isi{direct object}, e.g., “he had expressed himself extremely glad” [HCA 1/99/20], “he expressed himself sorry for it” [HCA 1/99/133], “he lamented himself under this condition very much” [HCA 1/99/70] “was bemoaning himself” [HCA 1/99/167], “gave orders himself” [HCA 1/99/72], “he says himself he was forced from the Sloop” [HCA 1/99/93], and “he says himself he put the Match to the Gun” [HCA 1/99/167]. This phenomenon may speak to a localized retention of prior \isi{transitive verb} structures (taking a \isi{reflexive pronoun} when the \isi{direct object} was the same as the subject) in sailors’ speech when most English speakers in the in Early Modern English were moving away from this pattern towards intransitive verbal expression in the same context (\citealt{MillwardHayes2012}: 263). Certain verbs, such as “behave,” were typically expressed with the older \isi{transitive} grammatical form in Ship English and the \isi{verb} invariably took a \isi{reflexive pronoun} when the \isi{direct object} was the same as the subject, e.g., “he haveing behaved himself so unjustly to them” [SP 42/6], “he behaved himself scandalously,” [SP 42/6], “behaved himself dutifully enough” [HCA 1/99/112], and “he hath behaved himself very diligently” (cited in \citealt{Brown2011}: 49). Other verbs such as “feel” also show the same expression as \isi{transitive} verbs with reflexive direct objects and similarly connect to the semantic field of personal expression, e.g., a ship doctor’s journal that records an interview with one of his patients “asking him when he was at stool, and how he feels himself” (cited in \citealt{Brown2011}: 48). Limited examples of this structure were evident with verbs that did not connect with the semantic field of personal expression, e.g., “[he] overslepte himself” [HCA 1/99/7] but most had some kind of link with oral or physical expression, suggesting that the retention of the older forms of \isi{transitive} verbal structures that permitted reflexives was conditioned by semantic rather than linguistic factors. 

In addition to their potential semantic conditioning with retained \isi{transitive} verbs, reflexive pronouns were also commonly used to stress human agency and intention in a context where so many sailors’ actions were restricted. Expressions associated with \isi{emphatic} agency commonly relate to sailors’ voluntary recruitment in opposition to the coercion of the press, e.g., “[he] shipped himself on board” [HCA 1/12/5], “he left her [his former ship] and shipt himself second mate and Gunner, on board of the Ship \textit{Succession}” [HCA 1/99 \textit{The American: Weekly Mercury} No.617, Oct 21-Oct 28 1731], “[he] Shipped himself” [HCA 1/13/97], “[he] Listed himself” [HCA 1/13/97], and “[he came] in order to enter himselfe on board” [HCA 1/53/68]. In the context of witness depositions and court records it was also important to mark sailors’ willingness in collaboration with pirates, and this was often done through the use of reflective pronouns to mark agency, e.g., “the whole fleet birthed themselves in their divisions \& moor’d” [ADM 52/2/6], “[the \isi{crew}] did as they wou’d themselves never observing him” [HCA 1/99/50], “through his means he made himself away” [HCA 1/11/110], “Capt Every but would have united himselfe with Capt Esq…” [HCA 1/53/14], and the partially-legible fragment “\& went himself in the ...” [HCA 1/53/32]\footnote{This last fragment is particularly interesting as “go” has rarely been used as a \isi{transitive verb} in English but there are reflexive varieties of the \isi{verb} “go” in Spanish and Portuguese (\textit{ir-se,} meaning to go away or leave,) suggesting that the adoption of this form might also speak to language context. It is important to note that some of these statements may also have been meant emphatically, an interpretation discussed in more detail below.} . Markers of agency were also important to stress sources of information in a largely oral culture of knowledge transfer, and this was often done using a \isi{reflexive pronoun} in a modifying \isi{prepositional phrase}, e.g., “he says of himself that…” [HCA 1/99/79], “he said little for himself” [HCA 1/99/42], “He say’d for himself that…” [HCA 1/99/28], and “he says for himself he has been only 5 Months with them” [HCA 1/99/165]. When used as an \isi{emphatic} marker of agency, a \isi{reflexive pronoun} is permitted in a \isi{post-nominal position} that re-asserts the \isi{noun phrase} subject, e.g., “Roberts the Commander of the Pyrate Ship, also himself told him…” [HCA 1/99/21], and “He himselfe had caused it to be done” [HCA 1/9/155], and “Robert Steewed himselve did resolve to be revenged upon the master if it cost him his life” [HCA 1/101/425]. The occurrence of a \isi{reflexive pronoun} in \isi{post-nominal position} features in the line “The sea itself on fire” in a \isi{seventeenth century} sea song (cited in \citealt{Palmer1986}: 67) and appears to attest to the salience of this type of structure in sailors’ speech. Reflexives were also permitted in non-restrictive modifying phrases, e.g., “he was flourishing his Cutlass... and cryed out ...\textit{himself being then just wounded}” [HCA 1/99/29, emphasis added], and “I am very much concerned to hear of any disorders committed at Kikotan, \textit{more especially my self}, which I am an utter strangor for” [CO 5/1411/654, emphasis added]. In sum, Ship English permitted reflexive verbs as markers of \isi{emphatic} agency in contexts in which reflexive pronouns were grammatically not required, notably as direct objects in verbs that may be expressed intransitively, in post-nominal positions that repeated the subject, and as part of modifying phrases. 

\subsection{{Relative pronouns}}\label{sec:5.4.6}

The word “which” is the most common \isi{relative pronoun} evident in the \isi{corpus} for both human and non-human referents. Non-human \isi{noun phrase} referents commonly attach a \isi{relative clause} headed with “which,” e.g., “saw a fleet to Cerousd which was our officers came from Plymouth” [ADM 52/1/1] and “The[y] called for the Pump handle, which was the instrument used to kill the rest” [HCA 1/99/9]. Human \isi{noun phrase} referents also commonly attach a \isi{relative clause} headed with “which,” e.g., “men which die yearly…the old Man which he found” [BL/74/816/m/11/36/2], “a woman which was a passenger” [HCA 1/101/372], and “two slaves which was purchased with Gun powder” [HCA 1/98/29]. Indeed, the \isi{relative pronoun} “which” is much more common in the \isi{corpus} when referring to human referents than the \isi{pronoun} “who” that was equally available to speakers of Early Modern English, e.g., “especially to the inhabitants, who, he that commends after the men of warr are departed has chiefly to deal withal” [SP 42/6]. Thus, we can surmise that for any \isi{noun phrase} referent, either non-human or human, the default \isi{relative pronoun} choice was “which” in modifying clauses and phrases. 

The \isi{relative pronoun} “which” could potentially eliminate the head \isi{noun} of an antecedent \isi{noun phrase} and thus appear to take a determiner, or could appear with a determiner in its own nominal construction. The \isi{relative pronoun} “which” sometimes appears to eliminate an antecedent \isi{noun head}, e.g., “those [people] wch came up to Day” [ADM 106/288/30], “he told one [person] which was nigh him” [HCA 1/99/59], and “three [ships] which gave us chase being french” [ADM 52/1/1]. In the three examples listed above, if the underlying \isi{noun} heads (assumed to be, “people,” “person,” and “ships”) are not realized then it might appear as if the \isi{relative pronoun} “which” is the \isi{noun} projection of the determiners “those,” “one” and “three” - an unlikely grammatical occurrence given that a \isi{relative pronoun} typically replaces an entire \isi{noun} phrases in any \isi{relative clause}. Instead, what seems to occur is that the use of the \isi{relative clause} permits the \isi{noun head} of the matrix phrase to be unrealized, and this may have been common in situations of oral communication when the referent was understood. Yet, the possibility of the \isi{relative pronoun} potentially taking a determiner in its own right, although anathema to modern-day speakers, was potentially acceptable to sailors of the early \isi{colonial period}. Examples include the quantifying determiner “all” in the letter excerpt, “to \textit{all which} I shall wait yor excys [your excellency’s] orders \& directions therein” [CO 5/1411/651 emphasis added], and the repeated examples of \isi{definite article} determiners in depositions such as “and upon \textit{the wch} day” [HCA 1/52/6, emphasis added], and journal excerpts, e.g., “several were kill’d on Shoar; \textit{the which} added much to my sorrow…they bad me go in; \textit{the which} I had not freedom to do…\textit{the which} I knew very well” [445f.1/20,27,36, emphasis added]\footnote{Similar rare examples are evident in historical usage. Evidence of “all” as a determiner features in the example “The Italian, French, and Spanish: \textit{all which} in a barbarous word” dated 1613. Additional evidence of “the which” in historical usage also shows the structure functioning as an \isi{adjective}, e.g., “the which copies” c1447, as a \isi{pronoun}, e.g., “the which I had almost forgot” c.1682, as a compound relative c. 1523, and as a \isi{pronoun} specific to people c. 1338 (\textit{Oxford Eng. Dict.} 1989, Vol 20: 224-225).}. Thus, whether the \isi{relative pronoun} created linguistic conditions in which the \isi{noun head} of the antecedent phrase could be unrealized, or if the \isi{relative pronoun} could take a determiner in its \isi{pronominal form}, it nonetheless created conditions in which it could acceptably follow a determiner. 

The word “which” could also function as a \isi{demonstrative} determiner with or without a repeated \isi{noun} antecedent. This feature occurs in court records when a witness is talking about some pistols and explains, “one of \textit{which pistolls} appears now” [HCA 1/99/7, emphasis added], it occurs again in the opening phrase of one day’s proceedings, “\textit{Which day} appeared personally Thomas Colston” [HCA 1/14/17 emphasis added], and it also occurs in published writing for a wider regional audience, “they met a Sloop at Sea from Boston bound to Maryland, the master of \textit{which sloop} came on board” [HCA 1/99 \textit{The American: Weekly Mercury} No.618, Oct 28-Nov 4 1731, emphasis added]. Two of the examples, specifically the record of court proceedings and the publication \textit{The American: Weekly Mercury,} feature in a formal context, potentially suggesting that the feature may have been conditioned by register. However, examples of the word “which” functioning as a determiner are few and do not suggest the kind of salience as evident in instances of the word functioning pronominally in relative constructions that are much more common in the (presumably) unplanned speech of witness depositions. 

When functioning as a \isi{relative pronoun} in the modifying clause of a \isi{noun phrase}, the word “which” may be deleted, even when it refers to the subject of the \isi{matrix clause}. Relative phrases functioning as nominal modifiers in which the relative subject \isi{pronoun} is omitted are evident in the \isi{corpus}, e.g., “to day came in a Brigginteens [which] have forced in from the Buoy” [ADM 52/2/5], and “this morning arrived here a hag boat from London [which] brought little or noe news at all” [ADM 52/1/8]\footnote{The default \isi{relative pronoun} “which” is provided in square brackets to indicate the underlying construction of the \isi{relative clause} from which the phrase is derived.}. Contemporaries identified the phenomena of omitted relative pronouns in sailors’ speech; repeated examples occur in the sea song “Lustily, Lustily,” e.g., “Here is a boatswain [which] will do his good will” (cited in \citealt{Palmer1986}: 3). It is notable that the example in this line of the song, a structure that is repeated in all three lines of the short stanza, indicates that even though it is the \isi{logical subject}, the \isi{relative pronoun} occurs not as in the position of the explicit subject of the sentence but in the complement position of the \isi{expletive} construction “Here is.” This very specific usage is also evident in the \isi{corpus}, e.g., “There are more yet [which] have been sent to lie at these places” [BL/74/816/m/11/36/3] in which the \isi{relative pronoun} fills the complement slot of the \isi{expletive} construction “There are.” Furthermore, it is not only in the context of logical subjects that relative pronouns are omitted from positions in the \isi{corpus}, there are also repeated instances of omitted relative pronouns that serve as modifiers to a \isi{noun phrase} functioning as \isi{direct object}, e.g., “we had a lighter [which] came aboard” [ADM 52/2/7], “Capt Fairbourne, to whom I delivered the letter [which] you sent me for him” [SP 42/6], and “nor knows any Body acquitted [which] can speak for him” [HCA 1/99/90]. Similarly, relative pronouns can be omitted when they serve as modifiers to a \isi{noun phrase} functioning as the object of a \isi{preposition}, e.g., “in the Manner [which] has been related” [HCA 1/99/105] and “he shifted himself in a shirt, [which] was not his own” [HCA 1/99/98]. There was no clear pattern of \isi{linguistic conditioning} that the \isi{corpus} examples indicated, yet it is possible that omission of the \isi{relative pronoun} may have been more common when strong verbs (i.e. irregular verbs in Modern English) featured in the relative construction, or when the relative construction included an \isi{auxiliary verb} of modality or aspect. 

\section{{Noun phrase modification} }\label{sec:5.5}

\subsection{{Types and placement of modifiers}}\label{sec:5.5.1}

Bare nouns and \isi{noun} phrases are frequently modified in the \isi{corpus} and one of the most common methods of modification is the addition of adjectives in \isi{pre-nominal position}, e.g., “\textit{Faire cleare} weather” [ADM 52/2/11, emphasis added]. Yet variation was permitted and modification using past and present participles as adjectives could be placed in pre- or post-nominal positions, e.g., (all with italic emphasis) “Severall \textit{arrived} men” [HCA 1/10/2], “a most dreadfull woman \textit{abused}” [HCA 1/101/429] and “two buts of beer \textit{Stinking}” [ADM 52/2/3].  In addition to \isi{participle} \isi{adjectival} forms, Ship English permitted \isi{noun} forms to take \isi{adjectival} function, e.g., (all with italic emphasis) “Anne Foster then \textit{Servant} maide” [HCA 1/101/429] “a twine or \textit{rope} yarn” [HCA 1/9/51], and “\textit{China} Ware which they took out of a \textit{China} Ship” [HCA 1/98/265]. Further supporting the interpretation that \isi{adjectival} words took \isi{nominal form} rather than \isi{adjectival} form is the fact that they can be pluralized, e.g (all with italic emphasis): “Contry \textit{Marchts} Ships” [ADM 52/2/5], “A colsultation [consultation] for \textit{flags} officers” [ADM 52/2/5], and “Showed his \textit{Hollands} Colours” [HCA 1/9/9]. Although the pluralization of nominal adjectives may derive from \isi{language contact}, it may also indicate an underlying \isi{genitive} function as in “Merchants’ ships” (the ships of the merchants,) “flags’ officers” (the officers of the flags,) and “Holland’s colours” (the colours of Holland). However, examples such as “In Company with \textit{others} Persons” [HCA 1/10/105, emphasis added] appear to suggest that the \isi{nominal form} “others” is pluralized in its own right and not as a derivation from an underlying \isi{genitive} form as it would imply “the persons of the others” rather than its most likely meaning “other people.” Thus, although it is not clear as to exactly why nominal forms, and specifically pluralized nominal forms, are permitted in \isi{pre-nominal position} with \isi{adjectival} function, the \isi{corpus} materials attest to their use in \isi{sailor}’s speech alongside variation in the placement of \isi{participle} adjectives either pre- or post-nominally. 

\subsection{{Present participle phrases} }\label{sec:5.5.2}

Speakers of Ship English make frequent use of modifying \isi{present participle} phrases. These \isi{present participle} phrases commonly follow a \isi{noun phrase} that they appear to modify, e.g., “\textbf{the wether} \textit{continuing hasy} stode sometimes of sometime on”\footnote{This and all subsequent examples include \isi{participle} phrases emphasized in italics and the adjacent \isi{noun} phrases emphasized in bold to draw attention to structure.} [ADM 52/1/1], “There \textbf{the winds} \textit{blowing hard} drove the ship to sea” [HCA 1/14/201], and “\textbf{he} \textit{standing hard}” [HCA 1/98/270]. However, \isi{present participle} phrases can also precede \isi{noun} phrases, e.g., “This day at 9 in the moring \textit{haveing feare weathr but Little wind we} came to saile” [T/70/1216/10] and “\textit{Haveing Soe conventient an opportunity I} thought my selfe bound to give your honor notice” [HCA 1/9/4] in which the \isi{present participle} phrases precede the pronominal subjects “we” and “I” respectively. Indeed, certain examples of orthography suggest that these phrases function as non-restrictive modifiers at the sentence level rather than functioning as embedded \isi{adjectival} \isi{noun phrase} modifiers, e.g., the parenthesis and the comma use in the excerpts, “which \textbf{the deponent} (\textit{showing some fear at}) he answered” [HCA 1/99/84] and “\textbf{William Batten} Master of the Experiment A Merchantship, \textit{now Liveing at Wapping}” [HCA 1/13/107]. The parenthesis and the comma usage in these examples suggest that the phrases are not bound within the \isi{noun} phrases “the \isi{deponent}” and “William Batten” but rather should be considered as non-restrictive modifiers with freedom to move into different locations in the sentence/utterance. Furthermore, the interpretation of such phrases as non-restrictive modifiers at the sentence level rather than nominal modifiers aligns with their common function as sentence adverbs. The adverbial nature of many of these types of structures is apparent through their frequency of \isi{stative} verbs, e.g., “there are \textbf{Publick Houses} \textit{scattering by the way-side}” [1045.f.3/1/18], “\textbf{Josiah Johnson} \textit{belonging to the same ship}” [HCA 1/99/69], “\textbf{all the men} \textit{now belonging to the said Burger}” [HCA 1/98/23], and “\textbf{he} \textit{having deserted her}” [HCA 1/99/165]. The number of phrases that also include non-essential but relevant background information also attests to the adverbial nature of the structure at the clause level, e.g., “\textbf{the Pyrates} \textit{wanting his water}” [HCA 1/99/110], “\textbf{the examinant} \textit{imagining that it was stolen goods}” [HCA 1/101/220], “\textbf{they} \textit{denying that they had sold any to him}” [HCA 1/9/53], “\textbf{Capt Warren} \textit{finding the Pink to sail heavy} left him” [HCA 1/98/267], and “\textbf{The Pyrates} \textit{having Seized their ship at Whydah January last}, forced those on Board” [HCA 1/99/77]. Thus, although the frequent present \isi{participle phrase} structures evident in the \isi{corpus} might all be interpreted as \isi{noun phrase} modifiers, they could also function at the sentence level by providing attendant circumstances, specifically in giving information that might be relevant in \isi{court testimony}. 

The significant number of \isi{present participle} structures in the \isi{corpus} is notable and these structures also appear to combine freely with other modifying phrases. The most common combination of modifying phrases happens with similar constituents, (i.e., another post-nominal present \isi{participle phrase}) and these occur with or without coordinating conjunctions or subordinating clause structures, e.g.,  “\textbf{The count} \textit{knowing him to be an old offender, having served the Company of Pyrates as Quarter Master till he was turned out by them} very readily found him Guilty” [HCA 1/99/36], “\textbf{the Pyrates}, \textit{hearing he was a Taylor}, and \textit{wanting Such a person very much}, did oblidge him” [HCA 1/99/28], “\textbf{they} \textit{having the command of his small Arms} and \textit{lying in the Great Cabin} and \textbf{he} \textit{having in the whole vessel but 12 men and boys}” [HCA 1/98/271], and “he had \textbf{former accuaintence} \textit{being page to earle of Winselsea who} \textit{being then in very good apparell having good store of mony} entertayned him” [HCA 1/101/408]. The post-nominal placement of the \isi{present participle} phrases in all of these examples serves as the common default, and this tendency is repeated when the correlating modifiers are prepositional phrases, \isi{past participle} phrases, or \isi{present participle} phrases, e.g., \textbf{Moody and the Cooper of the Pyrate [ship]} \textit{having their Pistolls about them} \& \textit{in search for that purpose} came to this…” ~[HCA 1/99/97], and “\textbf{A Servant of Masr John Smiths} \textit{called Richard the Tanner} was \textbf{one} \textit{haveing a light Candle}” [HCA 1/9/51]. In short, most modifying \isi{participle} phrases in the \isi{corpus} are post-nominal and although they may potentially embed into the \isi{noun phrase} as adjectives, many also combine freely with other modifying \isi{participle} phrases and prepositional phrases that also take an \isi{adverbial function} at the sentence level. 

There is some evidence that the present \isi{participle phrase} could function as the complete predicate of a clause although the main \isi{verb} lacks a finite. This phenomenon is most explicit when the subjects of adjacent clauses differ, leaving no possible interpretation that the \isi{participle phrase} is a modifier, e.g., “and \textbf{he} \textit{owning that he had 12 passengers which he brought from St. Marys} I slopp’d him and sent for mine” [HCA 1/98/270]. This example is a three-clause excerpt in which the first clause composes the subject “he” and the predicate “owning that he had 12 passengers” that has its own embedded \isi{relative clause} “which he brought from St. Marys.” In this example, the tense-less \isi{participle} “owning” appears to serve as the main \isi{verb} of the subject “he” in the \isi{matrix clause} and thereby suggests that \isi{present participle} structures could potentially assume main \isi{verb} status. In another example, the excerpt “\textbf{they} \textit{tarrying longer} the said Le Fort sailed away” [HCA 1/52/137] combines two clauses with the implication that the first clause “they tarrying longer” is subordinate to the second clause “LeFort sailed away” with a meaning that is equivalent to “…because they tarried [i.e., stayed/waited] longer, LeFort sailed away.” This example and others similar to it suggest that the clause formed with the \isi{present participle} was marked as subordinate to the \isi{matrix clause} that was expressed with a \isi{finite verb}. This interpretation that present participles compose clauses marked for \isi{subordination} is further supported by excerpts in which \isi{participle} phrases are coordinated with clauses in logbook entries, e.g., “\textbf{we} \textit{saying by our drums as the day before} and went for them” [T/70/1216/9]; they are also coordinated by parallel structures at the sentence level in court depositions, e.g., “we ware forsed to run to Anckor Againe \textbf{we} \textit{finding we Lost ground”} [T/70/1216/11]. The structure is even evident when the \isi{noun phrase} subject is omitted in logbook entries, e.g., “Saile with Keeft topsailes [we] \textit{haveing Showery Blowing weather}” [ADM 52/2/9]. In short, \isi{present participle} phrases could function as main verbs in a subordinating clause predicate in addition to their capacity to function as \isi{adjectival} \isi{noun phrase} modifiers and adverbial clause modifiers (comparable to Modern English,) and these potential realizations potentially contribute to the high frequency of the structure in the \isi{corpus}. 

\subsection{{Phrases headed with “being”}}\label{sec:5.5.3}

Present \isi{participle} phrases in a \isi{post-nominal position} and headed with the word “being” are a salient feature in the \isi{corpus} and appear in all registers and modalities sampled. The \isi{present participle} constructions follow the same usage patterns as identified in the previous paragraphs for all \isi{present participle} phrases, specifically: they can modify \isi{noun} phrases, e.g., “\textbf{the winds} \textit{being contrary}” [HCA 1/53/12]; they can function as non-restrictive modifiers at the sentence-level, e.g., “Then \textbf{it} \textit{Being high water} Slack Come to Saile” [ADM 52/2/9]; and they can function as predicates in subordinate clauses, either with or without an explicit \isi{nominal subject} e.g., “\textbf{he} \textit{being a Centry} beat one of their Ships Company” [HCA 1/99/139], and “they forced him [\textbf{he}] \textit{being unwilling}” [CO 5/1411/97]. However, the \isi{present participle} phrases headed with “being” are so salient in the \isi{corpus} that their usage deserves specific attention. For example, in a letter of just one page [HCA 1/98/254] four \isi{present participle} phrases occur and three of them are headed with the word “being:” first, “about three weeks after his arrival \textbf{his vessel} \textit{being fitted for his home ward bound voyage} then came in there the Beckford Gally,” second, “\textbf{she} \textit{being burnt by the Natives}” third, “\textbf{Capt Burges} \textit{being there} she got her passage,” and finally “\textbf{all the men} \textit{now belonging the Burges Shippe}” [HCA 1/98/254]. In one courtroom deposition, a speaker used three \isi{present participle} phrases in the same utterance to modify a single \isi{noun phrase}, two of which are headed by the word “being:” “\textbf{your prisoner} \textit{beinge a Marryner}, and \textit{goinge in the Good Shipp the Katherin of London}, and \textit{beinge beyond the Seas out of the Shipp} a difference fell between the \isi{prisoner} and one Richard Chubb” [HCA 1/101/209]. This profusion of \isi{present participle} phrases headed by the word “being” suggests that it was a preferred structure for expressing attendant details and \isi{stative} verbs in ship communities. 

There is evidence to suggest that the present \isi{participle phrase} headed by the word “being” was also used in a way that is comparable to modern day usage of linking verbs such as “appear” and “become” that can take a \isi{predicate noun} or predicate \isi{adjective}. In the \isi{corpus}, \isi{participle} phrases headed by “being” could take a \isi{bare noun} e.g., “\textbf{The 16 Day of} July \textit{being Monday}” [DDB6 8/4], “\textbf{a Discovery} \textit{being Death} to them” [HCA 1/99/92], and “\textbf{William Cornelius}…\textit{being skipper} in the said Shipp” [HCA 1/9/3]. And more complex \isi{noun phrase} predicates could include determiners, \isi{adjectival} components, adverbial components, and coordination e.g., “\textbf{one quarter} of what his Excelecy at present has, \textit{being two ounces of Cinnamon}” [CO 5/1411/654], “his only \textbf{Informations}, \textit{being old Slander}” [HCA 1/99/42], “\textbf{Tuesday} \textit{being the next day after Christmas day}” [HCA 1/52/1], “\textbf{he} \textit{being then Boatswain of her}” [HCA 1/99/75], and “Spoke with \textbf{them} att 7 in the morning \textit{being the Archangle \& two Mercht men}” [ADM 51/3797/6]. Comparable to modern \isi{linking verb} structures, \isi{participle} phrases headed by “being” could also take a bare \isi{adjective} predicate, e.g., .”..\textbf{Pond of fresh Water}… but \textit{being stagnant}, ‘tis only us’d for Cattle to drink” [1045.f.3/1/19], “\textbf{the cooke} \textit{being dead}” [HCA 1/101/405], and “At 6 last night \textbf{the flood} \textit{being made} we weighed [anchor]” [ADM 52/3/12]. However, the \isi{adjectival} constituents following a prepositional phrases headed by “being” are more commonly \isi{adjectival} phrases that included adverbial components, prepositional phrases with potential genitives and coordination, e.g., “\textbf{it} \textit{being late in the evening}” [T/70/1216/8] “\textbf{The English army} \textit{being landed and possessed of the Brass works}” [BL/Egerton 2395/0003], and “\textbf{the Commandr} of the said Vine pink \textit{being hindered of his Voyage}” [HCA 1/98/267]. Interestingly, the “being” phrase permits the use of another \isi{present participle} \isi{adjective} after the \isi{present participle} head (just like a modern \isi{linking verb} does) despite the doubling-up of present participles, e.g., “\textbf{wee} \textit{being rideing}” [ADM 52/2/1], “\textbf{his help} \textit{being wanting}” [HCA 1/99/109], “\textbf{depot} \textit{being then belonging}” [HCA 1/9/39], and “\textbf{The capt \& Lieut} \textit{being standing together}” [HCA 1/9/155]. In terms of frequency, the use of an \isi{adjectival} constituent (expressed as either an \isi{adjectival}, \isi{past participle}, or present \isi{participle phrase}) with the \isi{present participle} head “being” was the most common type of usage in the \isi{corpus} with nearly half of the constructions in a ninety-count sample showing collocation with \isi{adjectival} predicates (44 counts) compared to only around a third (29 counts) of collocation with \isi{noun} phrases (see \figref{fig:key:5.1}). Yet both combinations are common, and the versatility of the construction may explain why the present \isi{participle phrase} headed by the word “being” features so heavily in the \isi{corpus}. 

  
\begin{figure}
\begin{tikzpicture}
  [
      pie chart,
      slice type={Adjective}{lsMidBlue},
      slice type={Noun}{lsLightGreen},
      slice type={Adverb}{lsMidGreen}, 
      slice type={Infinitive}{lsLightBlue}, 
      pie values/.style={font={\small}},
      scale=2
  ]
  \pie{}{49/Adjective,32/Noun,18/Adverb,1/Infinitive}  
  \legend[shift={(2cm,7mm)}]{{Adjective (44)}/Adjective, Noun (29)/Noun,  Adverb (16)/Adverb, Infinitive (1)/Infinitive}
\end{tikzpicture}

% \todo[inline]{fix legend} 

\begin{tabular}{lrr}
\lsptoprule
Predicate & No. & Percent\\
\midrule 
Adjective & 44 & 49\\
Noun & 29 & 32\\
Adverb & 16 & 18\\
Infinitive & 1 & 1\\
\midrule 
	& 90 & 100\\
\lspbottomrule	
\end{tabular}

\caption{\label{fig:key:5.1}: Distribution of predicate types used with a sample of ninety present participle phrases headed by the word \textit{being}\\
{\tiny Sources: ADM 51/4322, 3797, HCA 1/101, \citealt{Brown2011}, T/70/1216, MMM BL/Egerton 2395, CO 5/1411, HCA 1/52, ADM 52/2, HCA 1/53, T/70/1/10, HCA 1/98, HCA 1/99, 1045.f.3/1, D/Earle/3/1, HCA 1/13, HCA 1/14, DDB6 8/4, HCA 1/9, \& HCA 1/10}
}
\end{figure}

 

In addition to the ability of the present \isi{participle phrase} headed by the word “being” to take predicate nouns and adjectives, it could also take predicate adverbs, specifically prepositional phrases that are adverbial in function. And this capacity draws parallels with the finite use of the \isi{copula} in modern English\footnote{The use of “being” as a \isi{copula} form is also implied in examples like “the Dutch were being about 60 Sayle of men of war” [ADM 52/2/5] in which the \isi{finite verb} “were” also takes a \isi{participle} “being” in an utterance with \isi{stative} meaning.} . Examples of such usage from the sample include: “\textbf{we} \textit{being in the trade wind}” [DDB6 8/4], “\textbf{they} \textit{being in a sloop}” [CO 5/1411/97], “\textbf{wee} \textit{being at an anker in 30 fathom}” [ADM 52/2/1], “so many \textbf{ships} \textit{being before me}” [D/Earle/3/1], and “\textbf{John Smith} the Msr \textit{being upon the quarterdeck”} [HCA 1/52/41]. Although the usage of the present \isi{participle phrase} headed by the word “being” with an adverbial predicate had the fewest counts in the sample (only 16 counts compared to 29 and 44 when used with predicate nouns and adjectives, respectively) the numbers were still significant with this type of usage representing nearly one fifth and 18 percent of the total constructions sampled (see \figref{fig:key:5.1}). Hence, the present \isi{participle phrase} headed by the word “being” could be considered as a kind of common \isi{linking verb} or \isi{copula} in Ship English, used with predicate nouns, adjectives and adverbs. 

Notable linguistic features associated with the present \isi{participle phrase} headed by the word “being” include the use of a \isi{genitive} \isi{pronominal form} preceding the phrase, the placement of adverbs before the predicate, and the tendency for the phrase to assume the function of a subordinate clause. The use of a \isi{genitive} \isi{pronominal form} (here emphasized in bold) preceding the “being” phrase seem to suggest that the phrase itself is nominal in function, e.g., “to justify \textbf{his} \textit{being forced}” [HCA 1/99/84], “since \textbf{his} \textit{being with the pirates”} [HCA 1/99/165], and “Haveing heard nothing before of \textbf{his} \textit{being dead}” [HCA 1/9/51]. This interpretation aligns with the use of the “-ing” word as a \isi{gerund} and is further supported by examples showing the phrase used as the object of a \isi{preposition} and as a \isi{nominal subject}, e.g., “told him nothing of \textbf{his} \textit{being to go with them} till the last Day” [HCA 1/99/114]\footnote{This example is also the only evidence of the \isi{infinitive} predicate used after the \isi{present participle} “being” that was found in the sample and composes the one percent of the graph in \figref{fig:key:5.1}.}  and “\textbf{his} \textit{not being suffered to go on Board of Prizes} makes him incapable of judging” [HCA 1/99/153].  Another feature of the “being” phrase is that it permits an adverbial particle (here emphasized in bold) after the \isi{present participle} and before the predicate, e.g., “Joseph Anderson \textit{being \textit{still} their} \textit{command}” [HCA 1/12/4], “the Spaniards men \textit{being \textit{all} in} \textit{the Army}” [HCA 1/99/9], “ago he \textit{being \textit{very} poor}” [HCA 1/13/97], and “he \textit{being \textit{so} Complaisant}” [HCA 1/99/132]. And, although not as common as one word adverbs, prepositional phrases functioning as adverbs are also permitted to occupy this space in the phrase, e.g., “[I] was very glad to see one of the King’s Ships, \textit{being \textit{before our coming} afraid} of Pyrates” [1045.f.3/1/23]. The “being” phrase also shows a tendency to assume the function of a subordinate clause, specifically showing a relationship between cause and effect that might be expressed with a conjunction like “because” or “so” in Modern English, e.g., “we were obliged [to trade] there \textit{being little sugar to be had”} [T 70/1/9], “we found the Stock broke we \textit{being forced to Cutt our best bowar} Cable” [ADM 52/2/5], “\textbf{this morning} \textit{being fair weather} we heeld \& scrubbd our Shipp” [ADM 52/2/6], and “but Doth not believe they will be able to fit her out againe \textbf{their} \textit{being so few men here}” [HCA 1/98/30]. The \isi{subordination} of the “being” phrase is explicit in the example “\textbf{Tho: Ashley, Stephen Bales and John Lother} \textit{being English men \& Prisoners of War in France} were taken on board” [HCA 1/13/98] in which the \isi{main clause} predicate “were taken on board” includes a \isi{finite copula}, which itself may have been expressed as “being taken on board” in a subordinate position. The complexity of such \isi{subordination} rules, in addition to the constraints that appear to govern \isi{genitive} pronominal collocation and the placement of adverbs before the predicate shows that the use of the present \isi{participle phrase} headed by the word “being” is conditioned by both linguistic and syntactic constraints in the \isi{corpus}. This ultimately contributes to the claim that sailors’ speech is characterized not only by the highly-popularized phonological assumptions connected with the profession and its technical jargon, but also by the frequency and constraints of specific words and nominal constructions that, combined, begin to show the bigger picture of Ship English as a comprehensive variety of the language.  

\section{{Summary} }\label{sec:5.6}

In terms of morphology, Ship English resists nominal compounding with markers of time and instead permits use of free morphemes with a range of nominal markers. Lexical distinctiveness is well represented in the literature on sailors’ speech but further research might expand our knowledge of how multilingual environments motivated lexical, morphological, and phonological \isi{feature transfer}. Ship English marks \isi{genitive case} with \isi{noun} compounding, \isi{possessive} pronominal forms, inflectional morphology, and prepositional phrases; and these variants occur concurrently in \isi{free variation}. Double \isi{genitive} marking occurs when a pronominal \isi{possessive} determiner is used in concordance with a \isi{prepositional phrase} and this is most common in idiomatic descriptions of rank that index a female vessel. Uninflected bare nouns are retained, even when marked as \isi{plural} by a cardinal determiner, and these frequently occur regarding units of distance, time, and weight measurement. Other \isi{uninflected} nominal forms referring to generic subjects may derive from words that were recent additions to the lexicon and had not yet undergone morphological regularization. The omission of \isi{nominal subject} heads occurs when the referent is understood from the context of the \isi{speech act} and alludes to shared knowledge, yet nominal head omission is also linguistically conditioned by the presence of a fronted \isi{participle}, or a prepositional or \isi{adverbial phrase}. Additionally, correlative clause constructions also permit the omission of the nominal head in the second clause. Object \isi{noun} phrases may be omitted in the \isi{direct object position} and when they appear as the object complement or the object of a \isi{preposition} and are potentially conditioned using an \isi{adjectival} modifier or the presence of pre-articles.

Demonstrative determiners were permitted to compete with other determiners in pre-nominal positions and \isi{accusative} pronouns could take a \isi{deictic function} when used pre-nominally. Ship English predominantly makes use of cardinal rather than ordinal numbers in pre-nominal positions but number agreement was not universal between determiner and \isi{noun} constituents, many of which remain \isi{uninflected} in \isi{plural} contexts. Sequential markers occur in predominantly post-nominal positions and although the distribution of temporal markers shows that specific lexemes are preferred in pre-nominal positions and for estimated time referents, the most common marker is “last.”  Categories of mass and count nouns are not mutually exclusive with various mass nouns having the capacity to be inflected for \isi{plural} nominal aspect and take a determiner specific to \isi{count noun} usage. Determiners specific to mass nouns are common in the context of \isi{indicative modality} yet specific usage appears to be governed by constraints that are not universal among the determiner class.  Articles are subject to significant localized \isi{free variation} and omission that is potentially conditioned by linguistic and semantic context in addition to stylistics common to the written mode.

The \isi{corpus} shows heavy use of pronominal forms including nominative, \isi{accusative} and \isi{genitive case} variants in a single phrase or clause, yet many instances of personal pronouns are redundant in idiomatic expression, double marked nominal phrases, prepositional phrases of agency, and when they function as grammatically unnecessary \isi{indefinite} objects. Possessive pronominal determiners commonly occur in collocation with \isi{gerund} phrases that express attendant circumstances and have the effect of building single complex clauses that are characteristic in the \isi{corpus}. Literate sailors used \isi{possessive} pronominal forms without specific anterior nominal markers in their personal communications, and logbooks show that pronominal expletives are preferred with tangible referents rather than being used to function as the grammatical subjects of existential constructions. Indefinite pronouns are not frequent in the \isi{corpus}, instead, Ship English marks \isi{indefinite} nouns using \isi{adjectival} particles that may compound with nominal forms but can also function as independent pronouns in \isi{free variation}. Reflexive pronouns were not semantically restricted to refer to the same subject referent of the clause in which they appear, nor were they grammatically restricted to the \isi{object position} of the clause. Used in variation, reflexives could occur in \isi{genitive case}, they could assume the subject position of a clause, and they were permitted in post-nominal modifying phrases. Their use was most common with certain verbs of personal expression where they were specifically used as markers of \isi{emphatic} agency. The word “which” is the most common \isi{relative pronoun} evident in the \isi{corpus} for both human and non-human referents. This \isi{pronoun} created conditions in which it could acceptably follow a determiner, although it is unclear whether this was the result of an antecedent \isi{noun head} being unrealized or if the \isi{relative pronoun} could take a determiner in its \isi{pronominal form}. The word “which” could also function as a \isi{demonstrative} determiner, potentially conditioned by \isi{genitive case} or formal register. Omission of relative pronouns is permitted when the relative construction modifies the object or the subject of the \isi{matrix clause}, but the conditioning for such omission is not clear from the \isi{corpus} examples. 

Noun phrase modification frequently occurs as a single-word \isi{adjective} in a \isi{pre-nominal position}, but nouns with \isi{adjectival} function and \isi{participle} adjectives can also modify \isi{noun} phrases and may occur in either pre- or post-nominal positions. The use of present \isi{participle phrase} modifiers occurs with salience and high frequency in the \isi{corpus} and, although these might all be interpreted as \isi{noun phrase} modifiers, they could also function at the clause level by providing attendant circumstances. Furthermore, when they assume an \isi{adverbial function}, these \isi{present participle} structures combine freely with other modifying \isi{participle} phrases and prepositional phrases that also take an \isi{adverbial function} at the sentence level. The most common type of present \isi{participle phrase} in the \isi{corpus} was headed by the word “being” which could be considered as a kind of common \isi{linking verb} or \isi{copula} in Ship English and was used with predicate nouns, adjectives and prepositional phrases that were adverbial in function. Usage of this present \isi{participle phrase} headed by the word “being” permits a \isi{genitive} \isi{pronominal form} to precede the phrase and permits the placement of adverbs before the predicate, and the phrase also shows a tendency to assume the function of a subordinate clause (albeit without a \isi{finite verb}) of a \isi{matrix clause} with a \isi{finite verb}. 

