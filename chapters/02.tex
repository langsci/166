\chapter{{Review of the literature}}

This chapter will begin with a summary of the work by the few scholars and enthusiasts who have recognized the importance of Ship English as a distinct and influential variety. This is followed by a more detailed presentation of studies on Ship English with a focus on the only two known published scholarly works with a focus on non-lexical characteristics of \isi{seventeenth century} \isi{sailor}’s English, namely, \citeapo{Matthews1935} monograph on pronunciation and \citegen{BaileyRoss1988} article on morphosyntactic features. The second part of this chapter will present a selected theoretical framework that underpins my own ideological stance and contextualizes the \isi{research design}. This framework is divided into a discussion of studies relating to \isi{dialect} change and \isi{dialect formation}, and an examination of some formative studies that have influenced my own thought process and the methodology for this research. 

\section{{Ship English: The work already done}}\label{sec:2.1}

\subsection{{Recognizing the importance of Ship English}}\label{sec:2.1.1}

Since Captain John Smith published \textit{Smith’s Sea Grammar} in 1627, the unique nature of sailors’ speech has been a popular subject of maritime training manuals and dictionaries for five centuries, as Bruzelius’ lists of maritime and naval dictionaries 17--19th century (1996; 1999; 2006) and the entry on ‘dictionaries’ in the \textit{Oxford Encyclopedia of Maritime History} \citep{Hattendorf2007} illustrate. And it is perhaps important to note that, in spite of the stereotyping present in fictional representations, there appears to be no stigma attached to learning this sea-language among occupational groups. Henry Manwayring states in the preface to his \textit{Sea-Man’s Dictionary} (1644) “this book shall make a man understand what other men say, and \textit{speak properly} himself” (emphasis added.) Even those accustomed to more courtly circles took efforts to learn how to speak “properly” in maritime contexts. For example, Samuel Pepys, Clerk of the Acts and Secretary of the Navy Board, promoted later to secretary of the Admiralty, bought a copy of Manwayring’s dictionary to learn the technical language of naval affairs. He notes in his diary (§March {1661}): “early up in the morning to read ‘the Seaman’s Grammar and Dictionary’ I lately have got, which do please me exceedingly well” (The National Maritime Museum, Samuel Pepys: Plague, Fire, Revolution, exhibit PBE 6233.)  This was just as well, because, like many other naval officers and administrators, “he had little experience of the maritime world, and no real qualifications for the job” (\citealt{Lincoln2015}: 144.) Speaking “properly” was therefore perhaps conducive to Pepys maintaining his position and generally reflective of the potential need of a whole group of administrators elected to their positions as a result of nepotism rather than experience. 

Administrators may have benefitted from manuals and dictionaries, but it was sailors themselves who learned though first-hand experience and were likely to have placed most value on the variety of speech native to their work and home environments, specifically, the use of a lexicon that constituted the \isi{professional jargon} of the \isi{crew}. In this respect, the fictional representation in Traven’s \textit{The Death Ship,} is likely accurate; the modern author describes how “each \isi{sailor} picks up the words of his companions, until, after two months or so, all men aboard have acquired a working knowledge of about three hundred words common to all the \isi{crew}” (1962: 237.) And it is most likely that the majority of such words were related to equipment, navigational or military techniques and routine aboard ship. For this reliance on a distinctive vocabulary, \citet{Hancock1986} describes Ship English as an “occupational \isi{dialect},” and Bailey \& Ross, recognize that “its lexical uniqueness is apparent” (1988: 207.) Shopen and Williams note that sailors commonly spread new lexical features around the ports they visit. For this reason, they refer to the importance of trade centers and shipping explicitly as factors that explain the linguistic changes that took place in the British Isles around the Middle English period (1980: 49--52.) Moreover, \citegen{Hickey2004} edited volume \textit{Legacies of Colonial English: Studies in Transported Dialects} additionally suggests that Ship English may have “incubated” new varieties of English that gave rise to dialects in places such as the United States, Australia and New Zealand (see \citealt{Hickey2004}: 50.) Hence, not only was the lexical uniqueness of sailors’ speech critical to the successful operation of the vessel, it may have also been critical in the formation of \isi{dialect} boundaries in the British Isles and potentially incubated overseas varieties.

Further to the impact that sailors potentially had in the formation of British dialects, \citet{Reinecke1938} was the first to claim that “the \isi{seaman} is a figure of the greatest importance in the creation of the more permanent makeshift tongues” (1938: 107.) He goes onto explain how sailors may have been pivotal in what linguists now refer to as the pidgin-\isi{creole} theory:

\begin{quotation}
Trade jargons may be regarded as the least developed forms of marginal language that have attained considerable fixity. Originally they arise out of the casual intercourse of traders (generally seamen) with a fixed population, although later they may be extended to serve the intercourse between the native population and resident foreigners (\citealt{Reinecke1938}: 110.) \end{quotation}

Subsequent scholars have echoed this claim, suggesting that maritime communities may have impacted the development of new languages derived from contact situations. For example, Hancock draws attention to the logic of Ship English serving as a hypothetical protoform in \isi{creole genesis}. He states, “Assuming a common origin for these Creoles, now spoken over 12,000 sea-miles apart, then the only possible historical link between them was the seamen and their speech” (\citealt{Hancock1976}: 33.) Since this early assertion in his 1976 paper “Nautical Sources of Krio Vocabulary,” Hancock has continued his work to evidence the role of mariners’ \isi{language use} in Krio, a \isi{creole} of Sierra Leone. Similarly, Holm’s extensive work on Nicaragua’s Miskito Coast \isi{Creole} identifies the importance of sailors as the agents of \isi{language contact} in his 1981 paper “Sociolinguistic History and the Creolist.” Both Hancock and Holm’s work influenced how subsequent scholars thought about the \isi{superstrate} in \isi{creole genesis theory}. In {1988} Bailey and Ross made the claim that sailors’ speech was the earliest form of English \isi{language contact} in many coastal regions around the Atlantic and Caribbean. Ship English therefore “seems to have been the earliest component of the \isi{superstrate}” in contexts of \isi{creole genesis} (\citealt{BaileyRoss1988}: 194.) They justify this statement by explaining that “sailors were instrumental in founding and maintaining the colonies where \isi{creole} languages developed” (\citealt{BaileyRoss1988}: 195.) Holm’s seminal text, \textit{Pidgins and Creoles,} published the same year as Bailey and Ross’s paper, echoes this statement: 

\begin{quotation}
Most Creoles arose in maritime colonies whose harbors docked slave ships, cargo ships, warships and countless smaller craft. Because of the mixture of dialects and even languages found among ships' crews, nautical speech has always constituted a distinctive \isi{sociolect}. (\citealt{Holm1988}: 78.) \end{quotation}

Holm’s theory that a \isi{creole} is an expanded \isi{pidgin} (1988: 7) in addition to the assertion that pidgins derive from \isi{language contact} with \isi{sailor}’s \isi{sociolect} in maritime colonies placed Ship English at the core of \isi{creole genesis} in studies leading up to the early 1980s. However, concurrently, there was a growing movement of substrate theories prompted by the second International Conference on \isi{Creole} Languages held at the University of the West Indies, Mona in April {1968} (\citealt{Hymes1971} §proceedings.) In the decades following this seminal conference, scholars of \isi{creole} studies began to explore the importance of West African languages that had been, up until this point, all but ignored in \isi{creole genesis theory}. The critical work of scholars such as Alleyne (1980; 1996,) Lefebvre (1986; 1998,) and \citet{Parkvall2000} led to a generally accepted idea that African substrates influenced \isi{creole} \isi{phonology}, syntax and semiotics whilst the \isi{superstrate} European languages became synonymous with the term ‘lexifier’ and a general belief that they predominantly contributed lexical forms.

Given the explicit association with \isi{superstrate} European languages and the term “lexifier” in \isi{creole} studies, it is perhaps not surprising that evidence to support the claim that Ship English impacted new varieties is mostly lexical. Holm observes, there is “an enormous amount of lexicon common to both sailors and Creoles” (1978: 98) and reinforces this in the description of entries in the \textit{Dictionary of Bahamian English} (\citealt{HolmWattSchilling1982}.) An example is the entry \textit{sound} which means to examine a person and derives from the nautical method to investigate the depth of water with a line and lead. Similarly, \citegen{CassidyLePage2002} \textit{Dictionary of Jamaican English} cites nautical etymology in a number of entries, e.g., the phrase \textit{chock and belay} which means tightly fastened and derives from a description of cargo that is perfectly and fully stowed. \citegen{Allsopp2003} \textit{Dictionary of Caribbean English Usage} lists 13 terms that are specifically traced to nautical origin and are used in regions from South-American Guyana, span the archipelago of the Caribbean, and reach as far as Central American Belize, e.g., \textit{kellick} used in Tobago, the Cayman Islands and Belize, which means a heavy stone and derives from the \isi{sailor}’s word for a small anchor. Although few, there are also studies that suggest \isi{language transfer} from maritime communities went beyond lexical items. For example, Lalla and D’Costa list 19 separate phonological features of maritime usage that are evident in eighteenth and nineteenth century Jamaican \isi{creole} (1990: 100) and Sullivan’s unpublished dissertation on \isi{pirate} counterculture in the Caribbean, and specifically the use of songs, shanties and chants that typify synchronized speech and unified work efforts, suggest that \isi{language transfer} was also happening at the discourse-level (2003: 458.) In sum, evidence shows that Ship English contributed to lexicon in Atlantic and Caribbean littoral regions and potentially impacted language features at all levels from the smallest phonological unit to the shaping of speech events, yet studies on features beyond the lexicon are few, most probably as a result of trends in \isi{creole} studies that associated European input with lexical influence.  

\subsection{{Studies on Ship English}}\label{sec:2.1.2}

Only two publications on Ship English, both based on ships’ logs, analyze features of the variety beyond its lexicon: \citet{Matthews1935} monograph on pronunciation, and \citegen{BaileyRoss1988} article on morphosyntactic features. Yet neither of these papers make strong claims about Ship English as a comprehensive variety. Matthews states in his introductory notes that what he presents:

should be regarded as a cross-section in the history of pronunciation, an account of the various pronunciations in use among the tarpaulin \isi{seaman} of the second half of the 17th century. It is not pretended that it describes the ‘\isi{seaman}’s \isi{dialect}’ of the period (1935: 196.)

Bailey and Ross conclude that “it is not at all clear that \textit{grammatically} Ship English is a unique \isi{sociolect}, although its lexical uniqueness is apparent” (1988: 207, authors’ italics.) The only other paper on Ship English since these early publications is an unpublished Master’s thesis \citep{Schultz2010} focusing on the sociolinguistic factors that caused the new variety to emerge, and, as a Master’s thesis, it includes no original research into the characteristic features of the variety itself. Hence, despite the many claims in the field that Ship English existed and was important in shaping \isi{dialect} boundaries in the British Isles and overseas, only two studies attempt an original analysis of non-lexical features that might have shaped \isi{language change} around the colonies and trading posts, and neither make very strong assertions about these features as representative of a comprehensive variety.  

Matthews’ monograph on \textit{Sailors’ Pronunciation in the Second Half of the 17\textsuperscript{th} Century} is an analysis of phonetic spelling in naval logbooks written between 1680 and 1700. The paper presents findings that describe “certain conventions of pronunciation for words used exclusively in the sea-trade” (1935: 13) and can thus be interpreted as indicative of general usage in wider maritime communities including aboard \isi{merchant} and privateer vessels, and in port communities. Matthews presents evidence in support of 67 apparent deviations from contemporary standard \isi{phonology} which are summarized below in terms of the phonological tendencies they reflect relating to vowels and consonants. 

Matthews’ findings on sailors’ pronunciation of vowels in the \isi{seventeenth century} indicate a tendency to raise certain vowels, for instance, /e/ is raised to [i,] particularly before a nasal consonant, e.g., \textit{twinty} ‘twenty’, \textit{frinds} ‘friends’ and \textit{pinquins} ‘penguins’ (\citealt{Matthews1935}: 200.) The vowel /u/ was also likely raised and fronted in a pronunciation that suggests the use of [ʌ] as a free variant, e.g., \textit{tuck} ‘took’, \textit{stud} ‘stood’, and \textit{luck} ‘look’ (209.) Other vowels are lowered; Matthews notes that [i] was subject to lowering and variation with [e,] illustrated in the words \textit{wech} ‘which’, \textit{seck} ‘sick’, and \textit{wend} ‘wind’ (199.) Matthews records variants between orthographic ‘a’, ‘e’ and ‘ea’, suggesting that they were realized as [e] or [ɛ,] e.g., \textit{fedem} ‘fathom’, \textit{Effreca} ‘Africa’, and \textit{leattar} ‘latter’ (201) and also notes a preference for unrounded variants in the realizations of the /ɔ/ phoneme. The two main variables that sailors appeared to use were [æ,] e.g., \textit{aspatall} ‘hospital’, \textit{last} ‘lost’, and \textit{shatt} ‘shot’, and [ʌ,] e.g., \textit{Hundoras} ‘Honduras’, \textit{stupt} ‘stopped’, and \textit{vulcano} ‘volcano’ (204--205.) Likewise, the realization of the lengthened /ɔ:/ phoneme also had an unrounded variant which Matthews concludes was probably [a:] based on the orthographic use of ‘a’ ‘aa’ and ‘ar’, e.g., \textit{sa} ‘saw’, \textit{straa} ‘straw’, and \textit{harse} ‘hawse’ (206.) 

Matthews’ findings on sailors’ pronunciation of consonants in the \isi{seventeenth century} shows a tendency towards \isi{free variation} in pairs of interchangeable phonemes, e.g., the interchange of /w/ and /v/ in words such as \textit{wery} ‘very’, \textit{winegar} ‘vinegar’, \textit{vayed} ‘weighed’, and \textit{avay} ‘away’ (\citealt{Matthews1935}: 235.) Alveolar and bilabial nasals are also both commonly interchanged, e.g., \textit{starm} ‘astern’, \textit{hamsome} ‘handsome’, \textit{inpressed} ‘impressed, and \textit{Novenber} ‘November’ (239.) Interchange of stop consonants involving the phonemes /k/, /t/, /d/ and /g/ are also evident (245,) and this interchange seems to be more dependent on whether the consonant is voiced or voiceless rather than dependent on the place of articulation, e.g., voiceless /k/ for voiceless /t/ in \textit{sleeke} ‘sleet’ and \textit{Lord Bartley} ‘Lord Berkeley’, and voiced /d/ for voiced /g/ in \textit{breidadeer} ‘brigadier’ (245.) Matthews observes that the phonemes /ŋ/, /θ/, /h/ and /w/ are commonly not pronounced in sailors’ speech of the \isi{seventeenth century}. The nasal /ŋ/ is often realized as [n,] particularly affecting final ‘-ing’ inflections as illustrated in the phonetic spellings of \textit{bearin} ‘bearing’, and \textit{lashens} ‘lashings’ (239) and /h/ is omitted in initial position, e.g., \textit{ospetall} ‘hospital’ and \textit{Obson} ‘Hobson’ and medial position, e.g., \textit{hogseds} ‘hogsheads’ and \textit{likleood} ‘likelyhood’ (230.) Similar omission of /w/ in initial and medial positions is illustrated by the examples \textit{ode} ‘wood’ and \textit{Westerds} ‘westwards’ (234.) Yet, contrary to consonant omission, Matthews finds that other consonants are intrusive or metathesize, for instance, the addition of [b] that frequently occurs after nasals in words such as \textit{Limbrick} ‘Limerick’ and \textit{Rumbley} ‘Romley’ (233) and the movement of [w] into word initial syllables, particularly after stops, e.g., \textit{dwoune} ‘down’ and \textit{twoer} ‘tower’ (235.)

Bailey and Ross’s article “The Shape of the Superstrate: Morphosyntactic Features of Ship English” (\citeyear*{BaileyRoss1988}) uses Matthews’ work as a starting point and extends the date range of his \isi{corpus} of naval logbooks from a twenty-year span between 1680--1700 to include all logs compiled up until 1725 and also the papers of the (British) Royal African Company. Their presentation of findings related to the morphosyntactic features of Ship English are qualified with the statement:

\begin{quotation}
Because the evidence from these sources is not easily quantifiable, our approach is necessarily inventorial, like that of creolists working with early historical records. We have attempted to document the presence of features that may have been influential in the evolution of Caribbean Creoles and BEV [Black English Vernacular] in the ships’ logs and to establish the constraints on their occurrence whenever possible. (\citealt{BaileyRoss1988}: 198.) \end{quotation}

Thus, the work of Bailey and Ross was explicitly influenced by methodology common to \isi{creole} studies. And their principal findings on \isi{verb} \isi{tense variation}, summarized below, were anticipated to have value in the scholarship of Caribbean \isi{creole} studies and African American \isi{dialect} studies of the United States.

Bailey and Ross’s findings relate principally to variation and constraints of \isi{verb} \isi{tense} realization in the present and past \isi{preterit} forms. They show that \isi{present tense} marking is realized in three ways, specifically by Ø, -s, or -th inflections. Yet, although all of these three inflections are common to Standard Early Modern English, the distribution of the inflections in Ship English differs from contemporary \isi{standard usage}\footnote{Note that the conjugations of verbs and the distributions of inflections were also variable across all English dialects.}. The Ø inflection occurs with all verbs except second person, e.g., with the \isi{third person singular} in “the Comondore [sic] who arrived here this Day and \textit{seem} to be very well pleased” (\citealt{BaileyRoss1988}: 199; this and all quotations from same source show authors’ italics.) The -s inflection more commonly occurs on verbs other than the \isi{third person singular}, e.g., with the first person singular in “I \textit{takes} it to the all Dutch forgeries” (199.) The -th inflection almost exclusively occurs with verbs that are \isi{third person singular} and is additionally constrained by the \isi{verb} used, e.g., with the \isi{third person singular} and the \isi{verb} LYE [lie] in “my Cheif [sic] mate \textit{Lyeth} desperately sick” (200.) Present \isi{tense} realizations of the \isi{verb} BE include \textit{is, are} and \textit{be}, with the \textit{is} realization predominating as a \isi{plural form} in the logbooks, e.g., “there \textit{is} some Traders” (201.) However, Bailey and Ross note that variation occurs from log to log and also within passages written by single individuals. 

Bailey and Ross observe that the very nature of the ships’ logs as a record of events provides an abundance of \isi{past tense} forms and conclude that “unmarked weak preterits (those without an <ed> or <t> suffix) are among the most common features of Ship English” e.g., “this day we \textit{kill} a Deare” (\citeyear*{BaileyRoss1988}: 202.) They also recognize that strong verbs, typically called irregular verbs in Modern English, also commonly had unmarked preterits in the logbooks, e.g., “Capt masters in ye Diana \textit{bring} a head” (203.) They additionally note that these unmarked strong preterits particularly occurred with certain verbs such as \textit{run, come, see, bring}, and \textit{got} (204.) However, strong verbs in the \isi{preterit} form were also potentially regularized, e.g., “we \textit{catched} at least 50” (204) or used as \isi{past participle} forms, e.g., “Captn Cooke \textit{has broke} his instructions” (204.) The \isi{verb} BE was realized most commonly in the logbooks as \textit{was} in both first and \isi{third person} subjects, singular and \isi{plural} compared to the comparative rarity of the word \textit{were} as a past realization (205.) Overall, and despite the range of options available to them, Bailey and Ross conclude that “The high frequency of unmarked verbs, both strong and weak, suggests that \isi{past tense marking} may have been optional for many speakers of Ship English” (205.) 

In addition to the majority of their findings on variations on how \isi{tense} is realized in \isi{verb} phrases, Bailey and Ross mention potential realizations of aspect and modality. They note that periphrastic DO may be a manifestation of aspect, e.g., “in this bay vessels \textit{doe} use to stop” (206) and the use of ‘like’ to mean ‘almost’ may be a manifestation of modality, e.g., “we...had \textit{like} to have taken” (206.) Yet these observations are limited to a few sentences supported by three examples and included in a miscellaneous section entitled “\textit{Other} morpho-syntactic features of Ship English” (emphasis added); wording that attests to the relative value that the authors placed on the observations of aspect and modality in \isi{verb} phrases. This miscellaneous section also includes lesser-observed features that affected \isi{noun} phrases such as unmarked plurals occurring with nouns of measure, e.g., “I see several \textit{saile} to windward” (205); \isi{relative pronoun} omission when functioning as subject and object, e.g., “there was a vessel came out of Fadm bound for Swanzey” (206); existential \textit{it}, e.g., “\textit{it} was very little wind” (206); and determinative \textit{them}, e.g., “ye Multitude of \textit{Them} foules” (206.) Yet these observations are likewise brief and conclude with a statement alluding to the complexity of determining their frequency. However, Bailey and Ross nonetheless recognize that “their presence does suggest that Ship English is likely to have included a number of relevant features that we simply cannot document” (206.) This statement, coupled with the last comment in the conclusion, that “While the inventory presented here is hardly an exhaustive account even of the morphosyntax of Ship English, it provides \textit{a place to begin}” (209, emphasis added) suggests that the authors were pointing to potential directions for future studies. However, since the publication of this paper in 1988, there have been no other studies published.  

\section{{Selected theoretical framework}}\label{sec:2.2}

\subsection{{Dialect change and new dialect formation} }\label{sec:2.2.1}

Johannes Schmidt’s \citeyear*{Schmidt1872} \textit{Wellentheorie} proposed the metaphor of waves starting from a single point in a pond to explain \isi{dialect} change. These waves could be of different strengths and concurrent with other waves that have different starting points, but the basic premise was that \isi{dialect} features spread in a pattern that is based solely on geographic adjacency. \citet{BaileyEtAl1993} later adapted the wave model by proposing that these waves of change could move through social space in addition to geographical space, and thus expanded Schmidt’s idea of adjacency to refer not only to geographical proximity, but also to social proximity (see \citealt{Petyt1980}: 50 and \citealt{AuerEtAl2005}: 7--9.) Nonetheless, the basic premise of the wave model and its geographical foci encourages assumptions about the obstruent nature of geographical features such as rivers and seas; yet according to Wakelin’s discussion of factors relevant to how variant \isi{dialect} forms emerge and are sustained: 

\begin{quotation}
As far as dialectal divisions are concerned, political and administrative boundaries appear to be of greater significance than geographical ones… the Thames, the Severn, the Tees and Tamar rivers, for example, do not seem to be important \isi{dialect} boundaries. Indeed, it is held that rivers (at least when navigable) act more often as a means of communication than as obstacles (\citealt{Wakelin1977}: 10.)\end{quotation}

Wakelin’s statement foregrounds social divisions rather than geographical ones, yet social models of \isi{dialect} change also use terms that perpetuate spatial associations and thus implicitly marginalize the potential influence of maritime communities. Many of these models integrate a concept of how linguistic innovations originate in “focal areas” that have cultural or political dominance, and which are also described as “places at the \textit{social center} of a language or \isi{dialect}” (\citealt{Tagliamonte2013}: 15, emphasis added.) Tagliamonte describes how \isi{language change} spreads from these “centers” by diffusion across populations from core areas to peripheral locations (\citeyear*{Tagliamonte2013}: 15.) The very words used to conceptualize these theories, namely, \textit{center, peripheral} and \textit{focal} encourage us to visualize the theory in spatial (and hence geographical) terms regardless of the context of the discourse that foregrounds social, political, and cultural factors. Consequently, this encourages us to discount the importance of littoral regions, as they are necessarily not “central;” thus we also marginalize the agency of maritime workers in this paradigm. A brief overview of these traditional models serves to illustrate perhaps one of the reasons that \isi{maritime language} communities have been excluded from consideration when investigating the factors that contribute to internal \isi{language change} in the field of dialectology. 

However, the role of sailors and maritime workers may have been pivotal to how \isi{dialect} zones formed and were maintained in an age before technological and flight networks formed new methods of contact. Historical dialectology provides evidence that \isi{dialect} boundaries cross bodies of water and that the presence of these bodies of water may indeed be the reason for the emergence of common features. For example, Tagliamonte’s \textit{Roots of English: Exploring the History of Dialects} (\citeyear*{Tagliamonte2013}) explains how, around the start of the \isi{seventeenth century}, south-west coastal Scotland and adjacent north-west coastal England had a common speech based on the Northumbrian \isi{dialect} of Old English with many shared Scots features. Features of this pan-coastal \isi{dialect} were then transported to coastal Northern Ireland by semi-transient maritime communities and were later reinforced by the speech varieties of settlers who moved from northern counties of England to the Ulster Plantations in Ireland at the beginning of the century. (\citealt{Tagliamonte2013}: 17.) Furthermore, Tagliamonte attests to a “pan-variety parallelism” across northern regions and across the Irish Sea in which “all communities share the same (variable) system in each case and it is only in the subtle weights and constraint of variation that the differences emerge” (\citeyear*{Tagliamonte2013}: 192.) This example suggests not only that water was no object to \isi{feature transfer}, but also that maritime communities may have served as hubs in communication networks that facilitated the transported linguistic features and established supra-regional norms. Although there has been no substantial research on the role of sailors in British \isi{dialect} zones, scholarship on the commonalities among coastal zones of the British Isles may provide key evidence for recognizing sailors as agents in the models and theories of \isi{language change} and \isi{new dialect formation}. 

Further to their agency in the shaping of \isi{dialect} zones in Britain, sailors may have also served a critical role in the development of overseas varieties. Thornton proposes that river and coastal trade routes, and hence also \isi{maritime speech} communities, were a prime factor in shaping the \isi{seventeenth century} Atlantic (\biberror{2000}: 56.) Moreover, beyond the Atlantic, the role of sailors as agents of \isi{language change} is recognized in \citegen{Hickey2004} \textit{Legacies of Colonial English: Studies in Transported Dialects.} Some theories presented in this edited collection have influenced how I conceptualize \isi{feature transfer} and \isi{language change} and, as such, are worth noting here. Wolfram and Schilling-Estes’ paper on “Remnant Dialects in the Coastal United States” has been particularly influential in the preliminary stages of my thinking about how new dialects might be formed through not only linguistic factors but also sociolinguistic and sociohistorical factors (\citeyear{WolframSchillingEstes2004}: 197.) This paper provided my model for an earlier study on the viability of \isi{seventeenth century} Pirate English as a distinct variety \citep{Delgado2013} and, as such, has been formative in my thinking about how Ship English may be considered as a distinct variety with characteristics features. Two other theories presented in Hickey’s edited volume have also influenced my thinking: firstly, “colonial lag,” also known as retention theory, in which variant features of modern day Englishes are directly attestable to differential input from the early contact situation \citep[8,]{Hickey2004} and secondly, a contrasting theory that contact dialects in early colonial situations may have had a more restricted role, namely, that they were “largely embryonic, providing incentives, starting points for future [regional] developments” (\citealt{Schneider2004}: 302.) 

Concurrent with Schneider’s work on “embryonic” language forms in the southern United States, \citegen{Trudgill2004} book \textit{New Dialect Formation: The Inevitability of Colonial Englishes,} published in the same year, develops his earlier theory of \isi{new dialect formation} as a result of mixing, \isi{leveling}, and simplification with a specific focus on Australian, New Zealand, and South African English varieties. Trudgill proposes that these new varieties of English were formed as a result of initial mixing among various regional British varieties in an isolated colonial territory that incubated the new form. The very fact that isolation is a factor in Trudgill’s model negates the presence of the maritime communities in contact with settlers and thus ignores their potential influence, yet this model of \isi{new dialect formation} has been influential in my own thinking and therefore deserves a closer examination. Trudgill describes the process of koineization in colonial territories in terms of its three stages: 1) \textit{mixing} of features results from a contact situation between variant regional and social dialects; 2) \textit{leveling} occurs when certain features are selected - or created from combining variants - and become the unmarked forms of the new \isi{speech community}, whilst at the same time there is a reduction or attrition of marked variants, and 3) \textit{simplification} happens with an increase in the morphophonemic, morphosyntactic and lexical regularity of the new standard forms (\citealt{Trudgill1986}: 90--103.)  Although Trudgill’s work on \isi{new dialect formation} explicitly relates to colonial English in the southern hemisphere, I anticipate that what he says is equally applicable to a variety incubated in maritime communities. His comments on the linguistic spectrum of the input speakers seem equally applicable to maritime workers as they do to New Zealand settlers: “\isi{dialect} mixture situations involving adults speaking many different dialects of the same language will eventually and inevitably lead to the production of a new, unitary \isi{dialect}...eventual convergence of order out chaos, on a single unitary variety” (\citealt{Trudgill2004}: 27.) Furthermore, what Trudgill claims about linguistic \isi{leveling} as a consequence of human desire for social conformity and group identification is equally applicable to sailors, and, as a result, his theory of mixing, \isi{leveling}, and simplification has particularly influenced how I have conceptualized the development of Ship English as a distinct variety\footnote{Although I argue here that Ship English was a distinct variety from other forms of speech, I also acknowledge the reality that all varieties of speech exist on a continuum and that non-standard varieties particularly develop out of a situation of pluri-lectal variation.} .

If, indeed, sailors incubated a new variety of English in their own communities, then it is entirely possible that this form was the one transported to new locations. An overview, and synthesis, of some of the literature that supports this interpretation follows. The premise that Ship English was a distinct type of speech derives from Bailey and Ross’s claim that it was “a changing and developing variety” (\citeyear*{BaileyRoss1988}: 207,) and Trudgill’s theory suggests that this may have been formed by the \isi{leveling} of other British regional and social dialects. Dobson’s work on Early Modern Standard English recognizes the formation of “a mixed \isi{dialect}, an amalgam of elements drawn from all parts of the country” (\citeyear{Dobson1955}: 35) that formed through a process of admixture that happened in England concurrent with the emergence of a Standard English. And, although there is no published scholarship on Ship English as a leveled variety, Schultz’s unpublished thesis claims that the development of Ship English by a process of \isi{dialect} \isi{leveling} was made possible by intensely consolidated and internally co-dependent maritime communities of practice, in which “linguistically, strong networks act as a norm enforcement mechanism” (\citeyear{Schultz2010}: 7--8.) Milroy’s article on social networks and linguistic focusing (\citeyear*{Milroy1986}) supports this interpretation, by referring back to Le Page’s theory that “the emergence of a closeknit group, a sense of solidarity and a feeling of shared territory are all conditions favouring [linguistic] focusing” (\citeyear*{Milroy1986}: 378.) My own earlier work on Pirate English \citep{Delgado2013} showed how one specific sub-community of mariners developed and maintained a distinct dialectal variety as a direct result of their networks of communication and consequent linguistic focusing. This idea of the existence of a new variety that was then transported overseas appears to be an interpretation supported by certain scholars working on \isi{pidgin} and creoles.  For example, Linebaugh and Rediker claim that “nautical English” as a distinct variety was one of the four inputs to Atlantic Pidgins along with Cant, Sabir, and West African languages (\citeyear{LinebaughRediker2000}: 153,) and Hancock claims that “it was this kind of English, an English having no single regional source in Britain, which the Africans first heard on their shores” (\citeyear*{Hancock1986}: 86.) Thus, although there is no single study attesting to the process of \isi{new dialect formation} in maritime communities, selected theories and observations in \isi{historical dialectology} support the premise. 

\subsection{{Formative studies influencing methodology}}\label{sec:2.2.2}

Laing and Lass, in their article \textit{Early Middle English Dialectology: Problems and Prospects}, identify as the major challenge of historical \isi{dialect} study the fact that “all of our informants are dead” (\citeyear*{LaingLass2006}: 418.) They claim that in this context, it is entirely feasible (and necessary) to base a research methodology on written sources, or what they describe as “text witnesses” of the contemporary dialects. These materials are then treated as if they were native speakers of the target \isi{dialect} and consequently, “take the place of informants who can be questioned directly” (418.) Thus, much of the following discussion of early English dialectology is based on linguistic suppositions derived from non-linguistic sources such as: colonial records \citep{Maynor1988}; reported speech, e.g., court records, depositions, executions, (\citealt{Awbery1988,Tagliamonte2013}); informal sources, e.g., letters, diaries \citep{Tagliamonte2013}; literary representations, e.g., songs, drama (\citealt{Russell1883,Wright1967}) and retrospectively compiled word lists (\citealt{Wright1967,Smith1627}.) These studies support and justify my own historical comparative approach that makes use of written source material to derive linguistic hypotheses about Ship English. 

Dublin’s Trinity College and the 1641 Depositions Project (Trinity College Dublin, MSS 809-841) is just one example of how transcribed spoken sources might be used for research. The database generated by the project maintains transcribed witness testimonies and depositions relating to the first-hand experiences of the 1641 Irish rebellion and can be searched by county, potentially facilitating investigators who might be interested in the linguistic features of a specific area. This \isi{corpus} of data and the observations of Laing and Lass on written sources serving linguistic research motivated my own focus on sailors’ depositions and witness testimony housed as part of the records of the Admiralty and Colonial State Papers at the National Archives, in Kew, London. 

Despite the availability of depositions in collections such as these, however, the limitations of written sources in linguistic research have, of course, been acknowledged in the literature. For example, in his chapter entitled “Written Records of Spoken Language: How Reliable Are They?” Maynor stresses that “even in the best of circumstances it is difficult for [such] dialectal research to be completely accurate” (\citeyear{Maynor1988}: 119) Given this caveat, the second aim of this research project, to generate baseline data, was formulated cautiously; I do not propose that my findings will form a comprehensive grammar of the \isi{dialect}, nor are they anticipated to escape critical comments from those who find the \isi{corpus} problematic. However, I believe that the aim of generating baseline data on the characteristic features of Ship English is reasonable and worthwhile given the limitations of the \isi{research design}. Furthermore, scholars of \isi{historical dialectology} who have chosen to investigate dialects of Old, Middle and Early-Modern English, or moribund and extinct varieties have used written evidence to document features and thus validate the necessity and value of using such a methodology in this study. 

\citegen{Lipski2005} \textit{A History of Afro-Hispanic Language} presents the findings of a study of reconstructed Afro-Hispanic speech over five centuries and spanning five continents. The aim of his extensive study is comparable to mine, in that Lipski investigates a marginalized speech variety that was often depicted with exaggeration and stereotype in the \isi{colonial period}, yet, he theorizes, has had a significant influence throughout the Spanish-speaking world. He also recognizes that the agency of Africans in Spanish \isi{language change} “is rarely considered on a par with more ‘traditional’ \isi{language contact} situations” (\citealt{Lipski2005}: 2.) The speech of sailors has likewise been neglected in decades of scholarship on \isi{language contact} and is often similarly depicted in exaggerated form with disdain or mockery when it is recognized as a distinct variety in non-academic and non-occupational writing. Similar to the varieties of Afro-Spanish that Lipski investigates, Ship English also has a limited and problematic \isi{corpus} of documented usage in addition to literary representations, second-hand reports and fragments of rhymes. As a result, Lipski’s comparative historical methodology served as an early model for my own preliminary studies. Specifically, his methodology influenced the \isi{research design} of my own pilot study on \isi{seventeenth century} sailors’ phonological forms, presented at The Society for \isi{Pidgin} and \isi{Creole} Linguistics Summer Meeting, University of Graz, Austria, 7--9 July {2015} in a paper entitled “The Reconstructed Phonology of Seventeenth Century Sailors’ Speech.” My \isi{research design} for this study compared Matthews’s phonological features of \isi{seventeenth century} sailors’ speech to representations in two texts: Defoe’s \textit{Robinson Crusoe} (\citeyear{Defoe1719}) and Johnson’s \textit{The Successful Pyrate} (\citeyear*{Johnson1713}) and concluded that the literary representations were valid linguistic records based on significant concordance with the historical data that Matthews observed in ships logs. This pilot study motivated the inclusion of shanties, fictional representations and third-party observations of \isi{sailor} talk in documents such as travel journals in my \isi{corpus}. Furthermore, in addition to the inclusion of literary documents and fragmentary data in his \isi{corpus}, Lipski’s ideological approach to linguistic analysis has also influenced my thinking. His analysis of linguistic data in conjunction with sociolinguistic data to present Afro-Hispanic language in human terms rather than a dispassionate list of features underpins the formation of my own \isi{research design} that integrates demographic and socio-historical data on speech communities in research on linguistic features.

Shaw includes demographic and socio-historical data in her study on \textit{Everyday Life in the Early English Caribbean: Irish, Africans, and the Construction of Difference} (\citeyear*{Shaw2013}). Although Shaw’s book is not linguistic in focus, she determines the characteristics of Irish and African community identity based on the implications in a range of data points cross-referenced with historical scholarship. Her research is comparable to mine in terms of the historical period of the populations in question and the geographical locations of their speech communities. It also analyses populations for whom we only have fragmentary and potentially biased documentation. Her findings are derived from “probing archival spaces and fissures” (190) and informed reconstruction around the data points that she has access to, and thus provides a further model for my own approach to a \isi{corpus} that includes fragmentary data. 

Comparable to Shaw’s book, \citegen{Jarvis2010} \textit{In the Eye of all Trade: Bermuda, Bermudians, and the Maritime Atlantic World {1680}-1783} contributes to an increasing body of historical scholarship that aims to present the complex lives of “largely anonymous individuals [who] shaped \isi{colonial expansion}” (459,) and his self-described maritime social history particularly succeeds in recognizing that maritime communities comprise more than the European-descended male figurehead that official documentation identifies. Jarvis explains that an extended kinship network was central to \isi{social cohesion} and this has motivated my own efforts to include non-Europeans, women, children and various other undocumented workers aboard ships and living in extended maritime communities in the scope of my own research. Jarvis’s introduction serves to highlight the importance of maritime movements to all interdisciplinary historical research: 

Motion was the defining characteristic of the Atlantic world. Connections and linkages across the space and central to all Atlantic histories. Whether the focus is people, plants, ideas, diseases, religious doctrines, texts, technologies, or commodities, crossing the water remains the assumed or explicit common denominator in most Atlantic studies. (\citealt{Jarvis2010}: 9.)

And although Jarvis does not include speech in his list of potential foci, linguistic studies around the Atlantic, and particularly at the time of early \isi{colonial expansion}, also depend on crossing the water in order to contextualize the patterns of \isi{feature transfer}, \isi{dialect} \isi{leveling}, and \isi{creole genesis} in littoral communities. Thus, Atlantic studies round out the interdisciplinary framework of my own research, in addition to \isi{historical dialectology}, socio-historical studies, and studies in pidgins and creoles that provide a comprehensive framework for my own investigation into Ship English of the early \isi{colonial period}. 

